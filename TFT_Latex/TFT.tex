\documentclass[12pt,a4paper]{report}
%\usepackage[utf8]{inputenc}
\usepackage[spanish]{babel}
\usepackage{amsmath}
\usepackage{amsfonts}
\usepackage{amssymb}
\usepackage{graphicx}
\usepackage[left=2.75cm,right=2.6cm,top=3.5cm,bottom=3.5cm]{geometry}



\usepackage{titletoc}
% Keywords command

\providecommand{\keywords}[1]
{
  \small	
  \textbf{\textit{Palabras clave:\hspace{0.3cm}}} #1
}




\usepackage[table]{xcolor}
\definecolor{naranja}{HTML}{E65113}
\usepackage[shortlabels]{enumitem}
\definecolor{slcolor}{HTML}{E65113}
\newcommand{\headlinecolor}{\color{slcolor}}
\usepackage{titlesec}

\definecolor{gray75}{gray}{0.75}
\newcommand{\hsp}{\hspace{-10pt}}

%%%%%%%%%%%%%%%%%%%%%%%%%%%%%%%%%%%%%%%%%%%%%%%%%%%%%%%%%%%%%%%%%%%%%%%%%%%%%
%------------------COMANDOS PARA EL TIPO DE LETRA---------------------------%
\usepackage{fontspec}
\usepackage[T1]{fontenc}
\usepackage{helvet}
\renewcommand{\familydefault}{\sfdefault}

\titleformat{\chapter}[hang]{\vspace{-3cm}\headlinecolor\Huge\bfseries}{\thechapter.\hsp}{20pt}{\Huge\bfseries}

%\titleformat{\section}[hang]{\Large\bfseries}{}{20pt}{\Large\bfseries}

\titleformat{\subsection}[hang]{\normalsize\bfseries}{}{20pt}{\large\bfseries}
\titleformat{\appendix}[hang]{\vspace{-3cm}\headlinecolor\Huge\bfseries}{\thechapter.\hsp}{20pt}{\Huge\bfseries}


%%%%%%%%%%%%%%%%%%%%%%%%%%%%%%%%%%%%%%%%%%%%%%%%%%%%%%%%%%%%%%%%%%%%%%%%%%%%%
%-------------------COMANDOS PARA TABLAS  E IMAGENES ---------------------%
\usepackage{tikz}
\usepackage{tabularx}

%%%%%%%%%%%%%%%%%%%%%%%%%%%%%%%%%%%%%%%%%%%%%%%%%%%%%%%%%%%%%%%%%%%%%%%%%%%%%
%-----------------COMANDOS PARA CABECERAS Y PIE DE PAGINA -----------------%
\usepackage{lastpage}
\usepackage{fancyhdr}
\usepackage{titlesec}

\fancypagestyle{plain}{%
  \renewcommand{\headrulewidth}{0pt}
\fancyhead{}
\fancyfoot{}
\fancyfoot[R]{{\scriptsize\thepage\ de \pageref{LastPage} | Título del TFT}}
\fancyhead[L]{\tikz[remember picture,overlay]\node[opacity=0.4] at (-3mm, 10mm){\includegraphics[scale=0.18]{./Images/image3.png}};}
\fancyheadoffset{0pt}
}

\pagestyle{plain}














%%%%%%-------------------+++++++++--INICIO DEL DOCUMENTO--+++++++++---------------------------%%%%%

\begin{document}



%------------------XXXX++++++ INICIO DE PORTADA  ++++++XXXXX-----------------%
\begin{titlepage}

\newgeometry{left=2.5cm, bottom=3cm, top=2cm, right=2.5cm}

\tikz[remember picture,overlay] \node[opacity=1,inner sep=0pt] at (73.6mm, -124.25mm){\includegraphics{./Images/Picture_TitlePage.jpg}};

{\fontfamily{phv}\selectfont
\fontsize{25}{10.4}\fontseries{b}\selectfont
\vspace{14cm}
\textbf{Desarrollo de un sistema de\\
diagnóstico de enfermedades\\
en hojas de tomate mediante\\ 
modelos de aprendizaje profundo}

\bigskip

\fontsize{12}{12}\selectfont
\fontseries{m}\selectfont
\vspace{5cm}
\centering
\begin{tabularx}{1\textwidth} { 
  || >{\raggedright}X 
  || >{\centering}X 
  || >{\raggedleft\arraybackslash}X || }
 Titulación:\\Máster en Big Data y Ciencia de Datos\\ 
 & Alumno/a: Marín Lucas, Rubén\\DNI: 07272889-J 
 & Convocatoria: \\
 Curso Académico\\ 2024-2025 
  & Director/a del TFT: Ricardo Lebrón Aguilar   
  & SEGUNDA  \\
\end{tabularx}
 }
\end{titlepage}
%--------------------XXXX++++++ FIN DE PORTADA  ++++++XXXXX-----------------%
\tableofcontents	
\addcontentsline{toc}{chapter}{\listfigurename}
\addcontentsline{toc}{chapter}{\listtablename}
\listoffigures
\listoftables

\begin{abstract}
Lorem ipsum (RESUMEN)
\vspace{0.5cm}

\keywords{primero, segundo, tercero}
\end{abstract}
\newpage
\section*{Agradecimientos}\label{sec:agradecimientos}
Lorem ipsum dolor sit amet, consectetur adipiscing elit, sed do eiusmod tempor incididunt ut labore et dolore magna aliqua. Ut enim ad minim veniam, quis nostrud exercitation ullamco laboris nisi ut aliquip ex ea commodo consequat. Duis aute irure dolor in reprehenderit in voluptate velit esse cillum dolore eu fugiat nulla pariatur. Excepteur sint occaecat cupidatat non proident, sunt in culpa qui officia deserunt mollit anim id est laborum.

	

\chapter{Introducción}\label{cap:cap1}

Tomate o tomatera (\textit{Solanum lycopersicum}) es una planta herbácea de la familia Solanaceae cultivada en todo el mundo para el cultivo de su fruto, el tomate o jitomate, uno de los ingredientes más universales de ensaladas y salsas en el mundo entero.

Según los últimos estudios filogenéticos, la planta silvestre de la cual surge el tomate doméstico actual tiene origen en la zona andina del norte de Perú y sur de Ecuador. Su domesticación y diversificación posterior se originó en México.

Los pueblos aztecas y mayas lo usaban en su cocina y fue exportado al resto del mundo a partir de la llegada de los españoles que lo distribuyeron a lo largo de sus colonias en el Caribe y la península ibérica a partir de lo cual pudo llegar al resto de eruopa. También lo llevaron a Filipinas y de allí pudo entrar al continente asiático.

\begin{figure}[ht]
\begin{center}
\includegraphics[scale=0.2]{./Images/Distribucion_tomate.jpg}
\caption{\headlinecolor{\underline{Origen del tomate}}}

\label{fig:fig1}

\end{center}
\end{figure}

El tomate es la hortaliza más extendida mundialmente y la de mayor valor económico. Su demanda aumenta continuamente y con ella su producción y comercio. 

La producción mundial de tomate ascendió a más de 186 millones de toneladas en 2022 segón los datos de la Organización de las Naciones Unidas para la Alimentación y la Agricultura (FAO). Según esta misma organización esta es la evolución de los 20 países que más han producido hasta 2022:

\begin{table}[ht] 
\centering
\caption{\headlinecolor{\underline{Top 20 países productores de tomates 2022}}}
\label{tab:my_label}
\begin{tabular}{|c|c|c|c|c|} \hline
\rowcolor{naranja}
\multicolumn{5}{|c|}{\textbf{Titulo}} \\ \hline
\rowcolor{naranja!30}
País & 2000 & 2010 & 2020 & 2022 \\ \hline
China & 22 200 & 46 760 & 64 680 & 68 242 \\ \hline
... & ... & ... & ... & ... \\ \hline
\end{tabular}
\end{table}

Como ya se ha mencionado, el cultivo de tomate es uno de los cultivos hortícolas más importantes a nivel mundial. Sin embargo, su producción se ve amenazada por una amplia variedad de enfermedades causadas por hongos, bacterias, virus y nematodos. Estas enfermedades pueden provocar una bajada de rendimiento que van desde reducciones parciales hasta la pérdida completa de la cosecha.

Entre las enfermedades más comunes se encuentran:

\begin{itemize}[label=$\bullet$]
\item Tizón tardío (\textit{Phytophthora infestans}): Puede destruir por completo una plantación si no se controla a tiempo, especialmente en condiciones húmedas y templadas.
\item Tizón temprano (\textit{Alternaria solani}): Produce defoliación progresiva, debilitando la planta y reduciendo el número y calidad de los frutos.
\item Fusariosis vascular (\textit{Fusarium oxysporum}): Ataca el sistema vascular, provocando marchitez y muerte de plantas.
\item Virus como TYLCV y TSWV: Pueden causar deformaciones severas y reducciones completas en la producción, especialmente cuando se transmiten por vectores como la mosca blanca.
\end{itemize}

La manifestación simultánea o sucesiva de estas enfermedades es una de las principales causas en la disminución en la productividad del cultivo a escala global. Además, muchas de estas enfermedades no solo viven en la planta sino que persisten en el suelo, semillas o herramientas que hayan inteactuado con la planta, lo que dificulta su erradicación y aumenta los costos del tratamiento.

Dada la magnitud del impacto de estas enfermedades, la detección temprana y precisa de las mismas es crucial. Permite una correcta intervención que minimiza las pérdidas, permitiendo la reducción del uso innecesario de los agroquímicos y mejorando la sostenibilidad. En este contexto, las tecnologías basadas en visión por computadora, sensores remotos e inteligencia artificial ofrecen soluciones eficaces para mejorar el seguimiento y el control sanitario de este cultivo clave.


\chapter{Objetivos}\label{cap:cap2}

El objetivo general de este proyecto consiste en conseguir un clasificador que a partir de imágenes de hojas de plantas de tomate distinga entre estado saludable y 10 enfermedades.

\section{Objetivos específicos}\label{sec:sec2.1}

\begin{enumerate}[label=\bfseries\headlinecolor\arabic*.]
\item Analizar dataset de hojas de tomate.
\item Implementar y entrenar modelos CNN para la clasificación.
\item Evaluar la precisión de los modelos y comparar resultados.

\end{enumerate}




%\includegraphics[scale=0.2]{./Images/image3.png}

\chapter{Estado del arte}\label{cap:cap3}

En los últimos años la aplicación de técnicas de inteligencia artificial en la agricultura ha cobrado un papel relevante, especialmente en tareas de diagnóstico temprano de enfermedades en cultivos.  El uso de aprendizaje profundo permite automatizar la detección de patrones en imágenes, lo cual puede ayudar a los agricultores a tomar decisiones más rápidas y eficientes.

Inicialmente, los métodos empleados para esta tarea incluían algoritmos de aprendizaje supervisado como máquinas de vectores de soporte (SVM), k\-vecinos más cercanos (KNN) y redes bayesianas. Sin embargo, estos enfoques dependían en gran medida de una segmentación previa precisa y de la extracción manual de características, lo que limitaba su capacidad de generalización y precisión en entornos reales.

Con la llegada de las redes neuronales convolucionales (CNN), se ha producido un cambio significativo en la forma de abordar este problema. Las CNN son capaces de aprender representaciones directamente a partir de los datos de imagen, eliminando la necesidad de ingeniería manual de características. Diversos estudios han demostrado su eficacia para la clasificación de enfermedades en hojas de tomate.

Completar con proyectos:

Maeda Gutiérrez (2019) realizó una comparativa entre varias arquitecturas CNN

Huerta Mora et al. (2024) emplearon la arquitectura VGG16 con técnicas de fine-tuning

propuesta colombiana presentada por DICYT (2023), donde se desarrolló un sistema de clasificación de enfermedades en tomate basado en MobileNet, capaz de ejecutarse en microcontroladores


\LaTeX{} \cite{latex2e} is a set of macros built atop \TeX{} \cite{texbook}.


\chapter{Implementación y desarrollo}\label{cap:cap4}


\chapter{Evaluación y resultados}\label{cap:cap5}


\chapter{Conclusiones}\label{cap:cap6}



\appendix
\chapter{Anexo I: Ejemplo de anexo}\label{cap:anexo1}
Lorem ipsum dolor sit amet, consectetur adipiscing elit, sed do eiusmod tempor incididunt ut labore et dolore magna aliqua. Ut enim ad minim veniam, quis nostrud exercitation ullamco laboris nisi ut aliquip ex ea commodo consequat. Duis aute irure dolor in reprehenderit in voluptate velit esse cillum dolore eu fugiat nulla pariatur. Excepteur sint occaecat cupidatat non proident, sunt in culpa qui officia deserunt mollit anim id est laborum.

\begin{enumerate}[label=\bfseries\headlinecolor\arabic*.]
\item Primer elemento.
\item Segundo elemento
\item Tercer elemento.
\begin{enumerate}[label=\alph*)]
\item Primer subelemento.
\item Segundo subelemento.
\begin{itemize}[label=$\bullet$]
\item Primer punto.
\item Segundo punto.
\end{itemize}

\end{enumerate}

\end{enumerate}

\chapter{Anexo II: Otro ejemplo de anexo}\label{cap:anexo2}
Lorem ipsum dolor sit amet, consectetur adipiscing elit, sed do eiusmod tempor incididunt ut labore et dolore magna aliqua. Ut enim ad minim veniam, quis nostrud exercitation ullamco laboris nisi ut aliquip ex ea commodo consequat. Duis aute irure dolor in reprehenderit in voluptate velit esse cillum dolore eu fugiat nulla pariatur. Excepteur sint occaecat cupidatat non proident, sunt in culpa qui officia deserunt mollit anim id est laborum.

Lorem ipsum dolor sit amet, consectetur adipiscing elit, sed do eiusmod tempor incididunt ut labore et dolore magna aliqua. Ut enim ad minim veniam, quis nostrud exercitation ullamco laboris nisi ut aliquip ex ea commodo consequat. Duis aute irure dolor in reprehenderit in voluptate velit esse cillum dolore eu fugiat nulla pariatur. Excepteur sint occaecat cupidatat non proident, sunt in culpa qui officia deserunt mollit anim id est laborum.

\bibliographystyle{plain}
%\bibliographystyle{alpha}
\bibliography{Bibliografia_TFT.bib}



\end{document}
