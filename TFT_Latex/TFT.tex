\documentclass[12pt,a4paper]{report}
%\usepackage[utf8]{inputenc}
\usepackage[spanish]{babel}
\usepackage{amsmath}
\usepackage{amsfonts}
\usepackage{amssymb}
\usepackage{graphicx}
\usepackage[left=2.75cm,right=2.6cm,top=3.5cm,bottom=3.5cm]{geometry}



\usepackage{titletoc}
% Keywords command

\providecommand{\keywords}[1]
{
  \small	
  \textbf{\textit{Palabras clave:\hspace{0.3cm}}} #1
}




\usepackage[table]{xcolor}
\definecolor{naranja}{HTML}{E65113}
\usepackage[shortlabels]{enumitem}
\definecolor{slcolor}{HTML}{E65113}
\newcommand{\headlinecolor}{\color{slcolor}}
\usepackage{titlesec}

\definecolor{gray75}{gray}{0.75}
\newcommand{\hsp}{\hspace{-10pt}}

%%%%%%%%%%%%%%%%%%%%%%%%%%%%%%%%%%%%%%%%%%%%%%%%%%%%%%%%%%%%%%%%%%%%%%%%%%%%%
%------------------COMANDOS PARA EL TIPO DE LETRA---------------------------%
\usepackage{fontspec}
\usepackage[T1]{fontenc}
\usepackage{helvet}
\renewcommand{\familydefault}{\sfdefault}

\titleformat{\chapter}[hang]{\vspace{-3cm}\headlinecolor\Huge\bfseries}{\thechapter.\hsp}{20pt}{\Huge\bfseries}

%\titleformat{\section}[hang]{\Large\bfseries}{}{20pt}{\Large\bfseries}

\titleformat{\subsection}[hang]{\normalsize\bfseries}{}{20pt}{\large\bfseries}
\titleformat{\appendix}[hang]{\vspace{-3cm}\headlinecolor\Huge\bfseries}{\thechapter.\hsp}{20pt}{\Huge\bfseries}


%%%%%%%%%%%%%%%%%%%%%%%%%%%%%%%%%%%%%%%%%%%%%%%%%%%%%%%%%%%%%%%%%%%%%%%%%%%%%
%-------------------COMANDOS PARA TABLAS  E IMAGENES ---------------------%
\usepackage{tikz}
\usepackage{tabularx}

%%%%%%%%%%%%%%%%%%%%%%%%%%%%%%%%%%%%%%%%%%%%%%%%%%%%%%%%%%%%%%%%%%%%%%%%%%%%%
%-----------------COMANDOS PARA CABECERAS Y PIE DE PAGINA -----------------%
\usepackage{lastpage}
\usepackage{fancyhdr}
\usepackage{titlesec}

\fancypagestyle{plain}{%
  \renewcommand{\headrulewidth}{0pt}
\fancyhead{}
\fancyfoot{}
\fancyfoot[R]{{\scriptsize\thepage\ de \pageref{LastPage} | Título del TFT}}
\fancyhead[L]{\tikz[remember picture,overlay]\node[opacity=0.4] at (-3mm, 10mm){\includegraphics[scale=0.18]{./Images/image3.png}};}
\fancyheadoffset{0pt}
}

\pagestyle{plain}














%%%%%%-------------------+++++++++--INICIO DEL DOCUMENTO--+++++++++---------------------------%%%%%

\begin{document}



%------------------XXXX++++++ INICIO DE PORTADA  ++++++XXXXX-----------------%
\begin{titlepage}

\newgeometry{left=2.5cm, bottom=3cm, top=2cm, right=2.5cm}

\tikz[remember picture,overlay] \node[opacity=1,inner sep=0pt] at (73.6mm, -124.25mm){\includegraphics{./Images/Picture_TitlePage.jpg}};

{\fontfamily{phv}\selectfont
\fontsize{25}{10.4}\fontseries{b}\selectfont
\vspace{14cm}
\textbf{Desarrollo de un sistema de\\
diagnóstico de enfermedades\\
en hojas de tomate mediante\\ 
modelos de aprendizaje profundo}

\bigskip

\fontsize{12}{12}\selectfont
\fontseries{m}\selectfont
\vspace{5cm}
\centering
\begin{tabularx}{1\textwidth} { 
  || >{\raggedright}X 
  || >{\centering}X 
  || >{\raggedleft\arraybackslash}X || }
 Titulación:\\Máster en Big Data y Ciencia de Datos\\ 
 & Alumno/a: Marín Lucas, Rubén\\DNI: 07272889-J 
 & Convocatoria: \\
 Curso Académico\\ 2024-2025 
  & Director/a del TFT: Ricardo Lebrón Aguilar   
  & SEGUNDA  \\
\end{tabularx}
 }
\end{titlepage}
%--------------------XXXX++++++ FIN DE PORTADA  ++++++XXXXX-----------------%
\tableofcontents	
\addcontentsline{toc}{chapter}{\listfigurename}
\addcontentsline{toc}{chapter}{\listtablename}
\listoffigures
\listoftables

\begin{abstract}
Lorem ipsum (RESUMEN)
\vspace{0.5cm}

\keywords{primero, segundo, tercero}
\end{abstract}
\newpage
\section*{Agradecimientos}\label{sec:agradecimientos}
Lorem ipsum dolor sit amet, consectetur adipiscing elit, sed do eiusmod tempor incididunt ut labore et dolore magna aliqua. Ut enim ad minim veniam, quis nostrud exercitation ullamco laboris nisi ut aliquip ex ea commodo consequat. Duis aute irure dolor in reprehenderit in voluptate velit esse cillum dolore eu fugiat nulla pariatur. Excepteur sint occaecat cupidatat non proident, sunt in culpa qui officia deserunt mollit anim id est laborum.

	

\chapter{Introducción}\label{cap:cap1}

Tomate o tomatera (\textit{Solanum lycopersicum}) es una planta herbácea de la familia Solanaceae cultivada en todo el mundo para el cultivo de su fruto, el tomate o jitomate, uno de los ingredientes más universales de ensaladas y salsas en el mundo entero.

Según los últimos estudios filogenéticos, la planta silvestre de la cual surge el tomate doméstico actual tiene origen en la zona andina del norte de Perú y sur de Ecuador. Su domesticación y diversificación posterior se originó en México.

Los pueblos aztecas y mayas lo usaban en su cocina y fue exportado al resto del mundo a partir de la llegada de los españoles que lo distribuyeron a lo largo de sus colonias en el Caribe y la península ibérica a partir de lo cual pudo llegar al resto de eruopa. También lo llevaron a Filipinas y de allí pudo entrar al continente asiático.

\begin{figure}[ht]
\begin{center}
\includegraphics[scale=0.2]{./Images/Distribucion_tomate.jpg}
\caption{\headlinecolor{\underline{Origen del tomate}}}

\label{fig:fig1}

\end{center}
\end{figure}

El tomate es la hortaliza más extendida mundialmente y la de mayor valor económico. Su demanda aumenta continuamente y con ella su producción y comercio. 

La producción mundial de tomate ascendió a más de 186 millones de toneladas en 2022 segón los datos de la Organización de las Naciones Unidas para la Alimentación y la Agricultura (FAO). Según esta misma organización esta es la evolución de los 20 países que más han producido hasta 2022:

\begin{table}[ht] 
\centering
\caption{\headlinecolor{\underline{Top 20 países productores de tomates 2022}}}
\label{tab:my_label}
\begin{tabular}{|c|c|c|c|c|} \hline
\rowcolor{naranja}
\multicolumn{5}{|c|}{\textbf{Titulo}} \\ \hline
\rowcolor{naranja!30}
País & 2000 & 2010 & 2020 & 2022 \\ \hline
China & 22 200 & 46 760 & 64 680 & 68 242 \\ \hline
... & ... & ... & ... & ... \\ \hline
\end{tabular}
\end{table}

(Enfermedades de planta de tomate y efectos en la producción)

...su producción se ve seriamente amenazada por una amplia variedad de enfermedades causadas por hongos, bacterias, virus y nematodos. Estas enfermedades no solo afectan la calidad del fruto, sino que también pueden provocar pérdidas de rendimiento que van desde reducciones parciales hasta el colapso completo de la cosecha...

Entre las enfermedades más comunes se encuentran:

... 

La incidencia simultánea o sucesiva de estas enfermedades es una de las principales causas de la disminución en la productividad del cultivo a escala global. Además, muchas de estas enfermedades persisten en el suelo, semillas o herramientas, lo que dificulta su erradicación y eleva los costos de manejo.

Dada la magnitud del impacto económico y agronómico, la detección temprana y precisa de enfermedades en plantas de tomate es crucial. Permite una intervención oportuna que minimiza pérdidas, reduce el uso innecesario de agroquímicos y mejora la sostenibilidad del sistema productivo. En este contexto, las tecnologías basadas en visión por computadora, sensores remotos e inteligencia artificial representan herramientas prometedoras para optimizar el monitoreo y la gestión fitosanitaria de este cultivo esencial.

\begin{figure}[ht]
\begin{center}
\includegraphics[scale=0.5]{./Images/image6.png}
\caption{\headlinecolor{\underline{Nombre de la gráfica 2}}}

\label{fig:fig2}

\end{center}
\end{figure}


\chapter{Prueba capítulo}\label{cap:cap2}

\section{Prueba sección}\label{sec:sec2.1}


\begin{enumerate}[label=\bfseries\headlinecolor\arabic*.]
\item Primer elemento.
\item Segundo elemento
\item Tercer elemento.
\begin{enumerate}[label=\alph*)]
\item Primer subelemento.
\item Segundo subelemento.
\begin{itemize}[label=$\bullet$]
\item Primer punto.
\item Segundo punto.
\end{itemize}

\end{enumerate}

\end{enumerate}

%\includegraphics[scale=0.2]{./Images/image3.png}

\chapter{Capítulo tercero}\label{cap:cap3}



\LaTeX{} \cite{latex2e} is a set of macros built atop \TeX{} \cite{texbook}.

\appendix
\chapter{Anexo I: Ejemplo de anexo}\label{cap:anexo1}
Lorem ipsum dolor sit amet, consectetur adipiscing elit, sed do eiusmod tempor incididunt ut labore et dolore magna aliqua. Ut enim ad minim veniam, quis nostrud exercitation ullamco laboris nisi ut aliquip ex ea commodo consequat. Duis aute irure dolor in reprehenderit in voluptate velit esse cillum dolore eu fugiat nulla pariatur. Excepteur sint occaecat cupidatat non proident, sunt in culpa qui officia deserunt mollit anim id est laborum.

Lorem ipsum dolor sit amet, consectetur adipiscing elit, sed do eiusmod tempor incididunt ut labore et dolore magna aliqua. Ut enim ad minim veniam, quis nostrud exercitation ullamco laboris nisi ut aliquip ex ea commodo consequat. Duis aute irure dolor in reprehenderit in voluptate velit esse cillum dolore eu fugiat nulla pariatur. Excepteur sint occaecat cupidatat non proident, sunt in culpa qui officia deserunt mollit anim id est laborum.
\chapter{Anexo II: Otro ejemplo de anexo}\label{cap:anexo2}
Lorem ipsum dolor sit amet, consectetur adipiscing elit, sed do eiusmod tempor incididunt ut labore et dolore magna aliqua. Ut enim ad minim veniam, quis nostrud exercitation ullamco laboris nisi ut aliquip ex ea commodo consequat. Duis aute irure dolor in reprehenderit in voluptate velit esse cillum dolore eu fugiat nulla pariatur. Excepteur sint occaecat cupidatat non proident, sunt in culpa qui officia deserunt mollit anim id est laborum.

Lorem ipsum dolor sit amet, consectetur adipiscing elit, sed do eiusmod tempor incididunt ut labore et dolore magna aliqua. Ut enim ad minim veniam, quis nostrud exercitation ullamco laboris nisi ut aliquip ex ea commodo consequat. Duis aute irure dolor in reprehenderit in voluptate velit esse cillum dolore eu fugiat nulla pariatur. Excepteur sint occaecat cupidatat non proident, sunt in culpa qui officia deserunt mollit anim id est laborum.

\bibliographystyle{plain}
%\bibliographystyle{alpha}
\bibliography{Bibliografia_TFT.bib}



\end{document}
