\documentclass[11pt,a4paper]{scrreprt}

% Gráficos y geometría
\usepackage{amsmath}
\usepackage{amsfonts}
\usepackage{amssymb}
\usepackage{graphicx}
\usepackage[left=3cm,right=3cm,top=4.4cm,bottom=2.19cm]{geometry} % márgenes 4.4 cm superior, 2.19 cm. inferior, izquierdo y derecho de 3 cm



% Índices y listas
\usepackage{titletoc}
\usepackage[shortlabels]{enumitem}
\newlist{bulletlist}{itemize}{1} % Definición de una lista personalizada llamada "bulletlist"
\setlist[bulletlist]{label=\LARGE\textbullet, left=1.3em}


% Colores
\usepackage[table]{xcolor}
\definecolor{naranja}{HTML}{E65113}
\definecolor{slcolor}{HTML}{E65113}
\newcommand{\headlinecolor}{\color{slcolor}}
\definecolor{gray75}{gray}{0.75}


\newcommand{\hsp}{\hspace{-10pt}}

% URLs e hipervínculos
\usepackage{url}
\usepackage{hyperref}

%%%%%%%%%%%%%%%%%%%%%%%%%%%%%%%%%%%%%%%%%%%%%%%%%%%%%%%%%%%%%%%%%%%%%%%%%%%%%
%------------------COMANDOS PARA EL TIPO DE LETRA---------------------------%
\usepackage{fontspec}
\usepackage[T1]{fontenc}
\usepackage{helvet}
\renewcommand{\familydefault}{\sfdefault}

% Estilo de capítulos y secciones con KOMA-Script
\usepackage{xcolor}
\setkomafont{chapter}{\Huge\bfseries\headlinecolor}
\setkomafont{section}{\Large\bfseries}
\setkomafont{subsection}{\normalsize\bfseries}
\setkomafont{sectioning}{\bfseries}




%%%%%%%%%%%%%%%%%%%%%%%%%%%%%%%%%%%%%%%%%%%%%%%%%%%%%%%%%%%%%%%%%%%%%%%%%%%%%
%-------------------COMANDOS PARA TABLAS  E IMAGENES ---------------------%
\usepackage{tikz}
\usepackage{tabularx}
\newcolumntype{Y}{>{\centering\arraybackslash}X} % centrado en columnas X

%%%%%%%%%%%%%%%%%%%%%%%%%%%%%%%%%%%%%%%%%%%%%%%%%%%%%%%%%%%%%%%%%%%%%%%%%%%%%
%-----------------COMANDOS PARA CABECERAS Y PIE DE PAGINA -----------------%
\usepackage{lastpage}
\usepackage{fancyhdr}

\fancypagestyle{plain}{%
  \renewcommand{\headrulewidth}{0pt}
\fancyhead{}
\fancyfoot{}
\fancyfoot[R]{{\scriptsize\thepage\ de \pageref{LastPage} | Título del TFT}}
\fancyhead[L]{\tikz[remember picture,overlay]\node[opacity=0.4] at (-3mm, 10mm){\includegraphics[scale=0.18]{./Images/image3.png}};}
\fancyheadoffset{0pt}
}

\pagestyle{plain}


% Palabras clave
\providecommand{\keywords}[1]
{
  \small	
  \textbf{\textit{Palabras clave:\hspace{0.3cm}}} #1
}

% Bibliografía APA
%\usepackage[utf8]{inputenc} % no hace falta con xelatex
\usepackage[spanish]{babel} 
\usepackage{csquotes}
\usepackage[backend=biber, style=apa]{biblatex}
\addbibresource{Bibliografia_TFT.bib}

% Interlineado y espaciado entre párrafos
\usepackage{setspace}
\setstretch{1.15} % interlineado 1.15
\setlength{\parskip}{0.8em} % espacio entre párrafos



%%%%%%-------------------+++++++++--INICIO DEL DOCUMENTO--+++++++++---------------------------%%%%%

\begin{document}



%------------------XXXX++++++ INICIO DE PORTADA  ++++++XXXXX-----------------%
\begin{titlepage}

\newgeometry{left=2.5cm, bottom=3cm, top=2cm, right=2.5cm}

\tikz[remember picture,overlay] \node[opacity=1,inner sep=0pt] at (73.6mm, -124.25mm){\includegraphics{./Images/Picture_TitlePage.jpg}};

{\fontfamily{phv}\selectfont
%\fontsize{25}{10.4}\fontseries{b}\selectfont
\fontsize{25}{25}\fontseries{b}\selectfont
%\vspace{14cm}
\vspace{12cm}
\textbf{Desarrollo de un sistema de\\
diagnóstico de enfermedades\\
en hojas de tomate mediante\\ 
modelos de aprendizaje profundo}

\vfill

\fontsize{12}{12}\selectfont
\fontseries{m}\selectfont
\vspace{4cm}
\centering
\begin{tabularx}{1\textwidth} { 
  || >{\raggedright}X 
  || >{\centering}X 
  || >{\raggedleft\arraybackslash}X || }
 Titulación:\\Máster en Big Data y Ciencia de Datos\\ 
 & Alumno/a: Marín Lucas, Rubén\\DNI: 07272889J 
 & Convocatoria: \\
 Curso Académico\\ 2024-2025 
  & Director/a del TFT: Ricardo Lebrón Aguilar   
  & SEGUNDA  \\
\end{tabularx}
 }
\end{titlepage}
%--------------------XXXX++++++ FIN DE PORTADA  ++++++XXXXX-----------------%
\tableofcontents	
\addcontentsline{toc}{chapter}{\listfigurename}
\addcontentsline{toc}{chapter}{\listtablename}
\listoffigures
\listoftables

\begin{abstract}
El cultivo de tomate es uno de los cultivos hortícolas más importantes a nivel mundial. Sin embargo, su producción se ve amenazada por una amplia variedad de enfermedades causadas por hongos, bacterias, virus y nematodos. Estas enfermedades pueden provocar una bajada de rendimiento que van desde reducciones parciales hasta la pérdida completa de la cosecha. Dada la magnitud del impacto de estas enfermedades, la detección temprana y precisa de las mismas es crucial. Permite una correcta intervención que minimiza las pérdidas, permitiendo la reducción del uso innecesario de los agroquímicos y mejorando la sostenibilidad. 

Los enfoques tradicionales para la detección de enfermedades suelen requerir mucha mano de obra, consumen mucho tiempo y son propensos a errores, lo que dificulta la intervención temprana. En este contexto el objetivo general de este proyecto consiste en desarrollar y evaluar un sistema inteligente basado en técnicas de aprendizaje profundo capaz de identificar el estado de salud y las principales enfermedades en hojas de tomate a partir de imágenes, asegurando un alto grado de precisión y capacidad de generalización.

Entre las tareas de preprocesamiento típicas se han transformado todas las imágenes a tamaños de 224x224 y se han normalizado. Además, se han usado las técnicas de \textit{data agmentation} y fijación de pesos en las clases con el objetivo evitar sesgos hacia ciertas clases y aumentar la capacidad de generalización. Posteriormente para el modelado se han creado modelos de redes neuronales convolucionales basados en las arquitecturas \textit{MobileNetV2}, \textit{EfficientNetB0} y \textit{NASNetMobile} preentrenadas con una base de datos de imágenes muy grande (\textit{ImageNet}), dado que permite aprovechar representaciones visuales previamente aprendidas como bordes, texturas y patrones para iniciar el entrenamiento en un dominio específico con menor cantidad de datos.

Entre los modelos evaluados, \textit{MobileNetV2} con 100 capas descongeladas (\textit{fine tuning}), tasa de aprendizaje (\textit{learning rate}) de \texttt{1e-5} y entrenado a 20 épocas, mostró el mejor rendimiento, alcanzando una precisión global del 94\% y un \textit{F1-score} macro de 0.939, lo que refleja una elevada capacidad de precisiób y una buena generalización entre las distintas enfermedades. 

Este resultado confirma la viabilidad del uso de redes neuronales convolucionales para la detección automática de patologías en hojas de tomate, ofreciendo una herramienta potencialmente útil para la monitorización temprana de cultivos y la reducción del uso de productos fitosanitarios, contribuyendo así a una agricultura más precisa y sostenible.

\keywords{detección de enfermedades, hojas de tomate, redes neuronales convolucionales, \textit{MobileNetV2}, \textit{fine tuning}}
\end{abstract}

\newpage
\section*{Agradecimientos}\label{sec:agradecimientos}
Quiero dar las gracias a toda mi familia y amigos por ayudarme y darme fuerzas para seguir adelante en todo momento.

Quiero agradecer a los amigos que he encontrado en en el máster, ya que me han ayudado a seguir adelante desde el comienzo. Seguramente alguna asignatura hubiera sido más difícil de superar sin vosotros.

Por último, gracias a todos los profesores que me he encontrado a lo largo de este viaje, ya que me han formado como el futuro ingeniero informático que seré.
	

\chapter{Introducción}\label{cap:cap1}

Tomate o tomatera (\textit{Solanum lycopersicum}) es una planta herbácea de la familia Solanaceae cultivada en todo el mundo para el cultivo de su fruto, el tomate o jitomate, uno de los ingredientes más universales de ensaladas y salsas en el mundo entero. \parencite{wikipedia1} 

Según los últimos estudios filogenéticos, la planta silvestre de la cual surge el tomate doméstico actual tiene origen en la zona andina del norte de Perú y sur de Ecuador. Su domesticación y diversificación posterior se originó en México. Los pueblos aztecas y mayas lo usaban en su cocina y fue exportado al resto del mundo a partir de la llegada de los españoles que lo distribuyeron a lo largo de sus colonias en el Caribe y la península ibérica a partir de lo cual pudo llegar al resto de eruopa. También lo llevaron a Filipinas y de allí pudo entrar al continente asiático. \parencite{agrotendencia1} 

\begin{figure}[ht]
\begin{center}
\includegraphics[scale=0.15]{./Images/Intro/Distribucion_tomate.jpg}
\caption{\headlinecolor{\underline{Origen del tomate}}}
\label{fig:origen_tomate}
\end{center}
\end{figure}

La producción mundial de tomate ascendió a más de 186 millones de toneladas en 2022 según los datos de la Organización de las Naciones Unidas para la Alimentación y la Agricultura (FAO). Según esta misma organización la evolución de los 10 países que más han producido hasta 2022 es la mostrada en el Cuadro \ref{tab:producción_mundial_tomate}. \parencite{wikipedia2}

\begin{table}[ht] 
\centering
\caption{\headlinecolor{\underline{Top 10 países productores de tomates 2022}}}
\label{tab:producción_mundial_tomate}
\begin{tabular}{|c|c|c|c|c|} \hline
\rowcolor{naranja}
\multicolumn{5}{|c|}{\textbf{Top 10 países productores de tomates 2022}} \\ \hline
\rowcolor{naranja!30}
País & 2000 & 2010 & 2020 & 2022 \\ \hline
China & 22 200 & 46 760 & 64 680 & 68 242 \\ \hline
India & 7430 & 12 433 & 20 550 & 20 694 \\ \hline
Turquía & 8890 & 10 052 & 13 204 & 13 000 \\ \hline
Estados Unidos & 12 622 & 14 053 & 10 939 & 10 200 \\ \hline
Egipto & 6786 & 8545 & 6494 & 6275 \\ \hline
Italia & 7538 & 6025 & 6248 & 6136 \\ \hline
México & 2666 & 2998 & 4137 & 4208 \\ \hline
Brasil & 3005 & 4107 & 3757 & 3810 \\ \hline
Nigeria & 1261 & 1800 & 3390 & 3685 \\ \hline
España & 3766 & 4313 & 4313 & 3652 \\ \hline
\end{tabular}
\end{table}

Como ya se ha mencionado, el cultivo de tomate es uno de los cultivos hortícolas más importantes a nivel mundial. Sin embargo, su producción se ve amenazada por una amplia variedad de enfermedades causadas por hongos, bacterias, virus y nematodos. Estas enfermedades pueden provocar una bajada de rendimiento que van desde reducciones parciales hasta la pérdida completa de la cosecha.

Entre las enfermedades más comunes se encuentran:

\begin{itemize}[label=\textbullet]
\item Tizón tardío (\textit{Late blight}): El agente causal es el oomiceto \textit{Phytophthora infestans} que afecta a las plantas solanáceas. Se desarrolla mejor en condiciones de humedad alta, lluvias, rocío persistente o días nublados, y temperaturas frías (entre 10\textdegree y 20\textdegree). Su ciclo incluye esporangios, zoosporas (en condiciones húmedas) y puede sobrevivir mediante estructuras resistentes (oosporas) o restos vegetales infectados \parencite{britannica}. Las primeras lesiones aparecen típicamente en hojas o tallos donde aparecen manchas de aspecto mojado que luego se tornan a manchas oscuras e irregulares. Bajo condiciones muy húmedas se observa una capa blanca o algodonosa de esporulación en el envés de las hojas u otras partes \parencite{winsconsinmadison} como se puede observar en la Figura \ref{fig:late_blight}.

\begin{figure}[ht]
\begin{center}
\includegraphics[scale=0.5]{./Images/Intro/Late-blight-tomato.jpg}
\caption{\headlinecolor{\underline{Tizón tardío en una planta de tomate}}} % https://www.purdue.edu/newsroom/archive/releases/2013/Q3/late-blight-disease-found-on-indiana-tomato-samples.html
\label{fig:late_blight}
\end{center}
\end{figure}

\item Tizón precoz (\textit{Early blight}): El agente causal principal es el hongo \textit{Alternaria solani} (y en algunos casos \textit{Alternaria tomatophila}) que puede sobrevivir en residuos de plantas enfermas, en el suelo, sobre plantas de la familia de las solanáceas y también transmitirse por semillas o plántulas infectadas \parencite{cornell}. Las condiciones favorables para su desarrollo incluyen: temperaturas cálidas (por encima de 25\textdegree), humedad elevada o periodos de hojas mojadas \parencite{pacificnorthwest1}. Los primeros síntomas suelen aparecer en las hojas más viejas, situadas en la base de la planta, aparecen manchas oscuras que a menudo presentan anillos concéntricos. Alrededor de esas manchas la hoja puede ponerse amarilla (clorosis) y luego necrosarse, lo que deriva en caída de hojas o defoliación parcial del vegetal \parencite{eorganic} como puede observarse en la Figura \ref{fig:early_blight}.

\begin{figure}[ht]
\begin{center}
\includegraphics[scale=0.2]{./Images/Intro/Early-blight-tomato.jpg}
\caption{\headlinecolor{\underline{Tizón precoz en una planta de tomate}}} % https://renaissancefarms.org/blog/early-blight-in-tomatoes/
\label{fig:early_blight}
\end{center}
\end{figure}

\item Mancha bacteriana (\textit{Bacterial spot}): Es causada por bacterias del género \textit{Xanthomonas} que atacan al tomate. El patógeno puede sobrevivir en residuos vegetales, en semillas contaminadas, en plantas voluntarias o en malas hierbas de la familia de las solanáceas \parencite{pacificnorthwest2}. Las condiciones favorables incluyen alta humedad y temperaturas templadas o calientes (20\textdegree o más). En las hojas aparecen manchas húmedas, luego se tornan oscuras, ásperas y con margen amarillo en ocasiones. Las manchas suelen aparecer primero en hojas viejas o en el dosel inferior de la planta, desde donde pueden subir \parencite{ucipm} como muestra la Figura \ref{fig:bacterial_spot}.

\begin{figure}[ht]
\begin{center}
\includegraphics[scale=0.2]{./Images/Intro/Bacterial-spot-tomato.jpg}
\caption{\headlinecolor{\underline{Mancha bacteriana en una planta de tomate}}} % https://www.growingproduce.com/crop-protection/a-new-bacterial-spot-strain-is-a-headache-for-tomato-growers/
\label{fig:bacterial_spot}
\end{center}
\end{figure}

\item Picadura de la araña roja de dos manchas (\textit{Two\-spotted spider mite}): El agente es la araña roja de dos manchas (\textit{Tetranychus urticae}) un ácaro (ni hongo, ni bacteria) que ataca muchas hortalizas incluyendo el tomate. Se alimenta mediante piezas bucales que perforan las células de las hojas (generalmente en el envés) y succionan el contenido, provocando daño celular, pérdida de clorofila y debilitamiento de la planta. Su reproducción es rápida, especialmente bajo condiciones de calor y sequedad, prefiere hojas viejas, la parte baja de la planta, y condiciones de ambiente seco, polvoso, con buena radiación y temperatura elevadas. En las hojas aparecen pequeñas manchas puntiformes, amarillas o blanquecinas, que resultan del daño celular tipo "picoteado" (\textit{stippling}) como se muestra en la Figura \ref{fig:two_spotted_spider_mite}. \parencite{umass}

\begin{figure}[ht]
\begin{center}
\includegraphics[scale=0.1]{./Images/Intro/Two-spotted-spider-mite.jpg}
\caption{\headlinecolor{\underline{Picadura de la araña roja de dos manchas en una planta de tomate}}} % https://ar.inspiredpencil.com/pictures-2023/red-spider-mites-on-tomato-plants
\label{fig:two_spotted_spider_mite}
\end{center}
\end{figure}

\item Virus de la hoja amarilla (\textit{Yellow leaf curl virus}): El agente causal es un virus los begomovirus, que afecta principalmente al tomate y ocasionalmente otras solanáceas. El vector principal es la mosca blanca, Bemisia tabaci, que adquiere el virus al alimentarse de plantas infectadas y luego lo transmite a plantas sanas. Las hojas muestran claramente: amarilleamiento entre las venas (clorosis interveinal), márgenes de la hoja amarillos, hojas enrolladas o curvadas hacia arriba y hacia dentro, y foliolos más pequeños de lo normal como muestra la Figura \ref{fig:yellow_leaf_curl_virus_tomato}. \parencite{diazpendon2010tylcv}

\begin{figure}[ht]
\begin{center}
\includegraphics[scale=0.13]{./Images/Intro/Yellow-leaf-curl-virus-tomato.jpg}
\caption{\headlinecolor{\underline{Virus de la hoja amarilla en una planta de tomate}}} % https://www.greenlife.co.ke/tomato-yellow-leaf-curl-virus/
\label{fig:yellow_leaf_curl_virus_tomato}
\end{center}
\end{figure}


\end{itemize}

La manifestación simultánea o sucesiva de estas enfermedades es una de las principales causas en la disminución en la productividad del cultivo a escala global. Además, muchas de estas enfermedades no solo viven en la planta sino que persisten en el suelo, semillas o herramientas que hayan inteactuado con la planta, lo que dificulta su erradicación y aumenta los costos del tratamiento. \parencite{agrotendencia2}

Dada la magnitud del impacto de estas enfermedades, la detección temprana y precisa de las mismas es crucial. Permite una correcta intervención que minimiza las pérdidas, permitiendo la reducción del uso innecesario de los agroquímicos y mejorando la sostenibilidad. En este contexto, las tecnologías basadas en visión por computadora, sensores remotos e inteligencia artificial ofrecen soluciones eficaces para mejorar el seguimiento y el control sanitario de este cultivo clave.


\chapter{Objetivos}\label{cap:cap2}

El objetivo general de este proyecto consiste en desarrollar y evaluar un sistema inteligente basado en técnicas de aprendizaje profundo capaz de identificar el estado de salud y las principales enfermedades en hojas de tomate a partir de imágenes, asegurando un alto grado de precisión y capacidad de generalización.

\section{Objetivos específicos}\label{sec:sec2.1}

\begin{enumerate}[label=\bfseries\headlinecolor\arabic*.]
\item Revisar el estado del arte relacionado con la detección de enfermedades en plantas, con espacial énfasis en el cultivo del tomate. 
\item Analizar y preparar el conjunto de datos aplicando las técnicas necesarias de preprocesamiento de imágenes para garantizar la calidad y homogeneidad del material de entrenamiento.
\item Implementar y entrenar distintos modelos utilizando arquitecturas preentrenadas con alguna base de datos para adaptarlas a la tarea de clasificación de enfermedades en hojas de tomate. 
\item Evaluar los modelos desarrollados comparando su rendimiento mediante métricas como precsión, sensibilidad o capacidad de generalización.
\end{enumerate}


\chapter{Estado del arte}\label{cap:cap3}

En los últimos años la aplicación de técnicas de inteligencia artificial en la agricultura ha cobrado un papel relevante, especialmente en tareas de diagnóstico temprano de enfermedades en cultivos.  El uso de aprendizaje profundo permite automatizar la detección de patrones en imágenes, lo cual puede ayudar a los agricultores a tomar decisiones más rápidas y eficientes.

Inicialmente, el enfoque tradicional para las tareas de clasificación de imágenes se basaba en características diseñadas manualmente como SIFT \parencite{research1}, HoG \parencite{research2}, SURF \parencite{research3}, etc, para luego usar algún algoritmo de aprendizaje supervisado como máquinas de vectores de soporte (SVM), k\-vecinos más cercanos (K\-NN) y redes bayesianas. Sin embargo, estos enfoques dependían en gran medida de las características predefinidas. La ingeniería de características en sí misma es un proceso complejo y laborioso que debe revisarse cada vez que el problema en cuestión o el conjunto de datos asociado cambia considerablemente. Este problema se da en todos los intentos tradicionales de detectar enfermedades de las plantas , ya que se basaban en gran medida en características diseñadas manualmente, técnicas de mejora de imágenes y otras metodologías complejas.

Además, los enfoques tradicionales para la clasificación de enfermedades mediante \textit{machine learning} suelen centrarse en un número reducido de clases, normalmente dentro de un mismo cultivo. Por ejemplo, se han usado imágenes térmicas y estéreo para detectar el moho polvoriento (\textit{powdery mildew}) en hojas de tomate \parencite{research4}, imágenes RGBD para detección del tizón del manzano (\textit{apple scab}) \parencite{research5} y sensores basados en aeronaves para la detección del virus del rizado amarillo del tomate (\textit{tomato yellow leaf curl}) mediante el uso de un conjunto de pasos clásicos de extracción de características \parencite{research6}.

Posteriormente han aparecido estudios que usan entrenamiento supervisado de extremo a extremo utilizando una arquitectura de red neuronal convolucional como \parencite{research7} que demostraron que este tipo de arquitecturas tienen un buen desempeño incluos en problemas de clasificación de imágenes con un número de clases, superando con creces a los enfoques tradicionales. El hecho de no depender de la laboriosa fase de la ingeniería de características, además de la generalización que alcanza convierte este enfoque en un candidato prometedor para la detección computacional de enfermedades en plantas. \parencite{research8}

Para aplicaciones de diagnóstico de enfermedades en plantas es habitual seleccionar arquitecturas que equilibren precisión y coste computacional. Arquitecturas ligeras como \textit{MobileNetV2} y \textit{NASNetMobile} han demostrado ser adecuadas para sistemas móviles por su bajo número de parámetros y alta velocidad de inferencia, manteniendo precisiones comparables a modelos más pesados tras \textit{fine tuning} \parencite{research9}. Por su parte, \textit{EfficientNetB0}, diseñado mediante \textit{compound scaling}, ofrece una eficiencia de parámetros superior y suele vencer a tradicionales en relación precisión y tamaño \parencite{research10}. Estas características hacen que \textit{MobileNetV2}, \textit{EfficientNetB0} y \textit{NASNetMobile} sean opciones sólidas para aplicaciones prácticas de diagnóstico en tomate, especialmente cuando el despliegue en dispositivos con recursos limitados.

El uso de modelos preentrenados en bases de datos extensas como \textit{ImageNet} constituye una práctica consolidada en tareas de clasificación de enfermedades foliares, dado que permiten aprovechar representaciones visuales previamente aprendidas (bordes, texturas, patrones) para iniciar el entrenamiento en un dominio específico con menor cantidad de datos. Estudios recientes demuestran que el uso de \textit{transfer learning} a partir de arquitecturas entrenadas sobre \textit{ImageNet} acelera la convergencia, reduciendo el número de épocas (\textit{epochs}), cantidad de datos y riesgo de sobreajuste (\textit{overfitting}) \parencite{research11}. No obstante, también se advierte que la diferencia de dominio entre \textit{Imagenet} y las imágenes reales de hojas puede afectar a la generalización, lo cual resalta la importancia de una augmentación adecuada, ajustes de hiperparámetros y validación con imágenes de campo \parencite{research12}.

En este trabajo se emplean modelos basados en las aruitecturas \textit{MobileNetV2}, \textit{EfficientNetB0} y \textit{NASNetMobile} preentrenados con \textit{ImageNet} como punto de partida, lo que permite centrar el esfuerzo en la adaptación al problema de enfermedades de tomate mediante fine-tuning y validación rigurosa.

En la literatura sobre clasificación automática de enfermedades foliares mediante redes neuronales convolucionales, la evaluación del rendimiento del modelo rara vez se limita a la precisión global. Se emplean métricas complementarias como precisión, sensibilidad (\textit{recall}) y F1-Score para reflejar la capacidad del modelo tanto de minimizar falsos positivos como falsos negativos \parencite{}, lo cual es crítico en un contexto agronómico donde tanto detectar tarde una enfermedad como etiquetar erróneamente una hoja sana pueden acarrear consecuencias importantes. De la misma manera, es interesante analizar la función de pérdidas de validación durante el entrenamiento, ya que permite evaluar la convergencia y ayudar a la detección de sobreajuste \parencite{}.

Por otra parte, no siempre es suficiente con observar métricas sobre el modelo en su conjunto, sino que a veces también es relevante analizar métricas como las mencionadas (precisión, sensibilidad y F1-Score) por clases, para determinar si existe algún sesgo hacia alguna clase \parencite{}. En esta línea para apoyarse en alguna métrica más visual también es interesante el uso de la matriz de confusión que permite ver cómo se comporta el modelo entre diferentes clases pudiendo visualizar el número de imágenes por clases, los predicciones sobre las mismas con sus aciertos y fallos de un vistazo \parencite{}.

Más allá de entrenar una única red convolucional sobre imágenes de hojas, los estudios recientes incorporan un enfoque adicional que consiste en crear \textit{ensembles} o las combinaciones de diferentes arquitecturas y luego combinan sus predicciones por votación (\textit{hard-voting}, \textit{soft-voting}) o mediante promedios ponderados. En esta línea nos encontramos estudios como el realizado en el artículo que utilizaron tres arquitecturas, \textit{DenseNet201}, \textit{InceptionV3} y \textit{Xception} y un método de \textit{weighted soft-voting}  para lograr un precisión superior al de cada modelo individual \parencite{}.


\chapter{Implementación y desarrollo}\label{cap:cap4}

En este capítulo se presenta tanto el \textit{hardware} como el \textit{software} usados en este proyecto. Además se explica la procedencia y estructura del conjunto de datos que serán usados para el estudio. Finalmente, se desarrolla el preprocesmiento que se realiza a este conjunto de datos junto con los modelos entrenados para conseguir un clasificador.

\section{Herramientas usadas}\label{sec:sec4.1}

Para llevar a cabo este proyecto, se ha usado Google Colab (abreviatura de Google Colaboratory) que se accedía desde el ordenador portátil del autor del documento. Este ordenador es un ASUS TUF Gaming FX505GT que cuenta con las siguientes características:

\begin{bulletlist}
\item 16 GB de RAM con formato DDR4.
\item Almacenamiento compuesto por un disco duro con tecnología SSD de 512GB.
\item Procesador Intel Core i7-9750H CPU a 2.60 GHz, con 6 procesadores principales y 6 procesadores lógicos.
\item Tarjeta gráfica NVIDIA GeForce GTX 1650 con 4GB de RAM.
\end{bulletlist}

Google Colab es un servicio gratuito de Google que permite escribir y ejecutar código en la nube sin necesidad de instalar nada en tu equipo. Los recursos que ofrece de forma gratuita varían con el tiempo, pero las características que suele ofrecer son las siguientes:

\begin{bulletlist}
\item GPU NVIDIA Tesla T4 con 16 GB de VRAM ó CPU Intel Xeon con alrededor de 13 GB de RAM.
\item Almacenamiento temporal se corresponde con unos 100 GB de espacio en disco.
\item La duración de la sesión puede ser de hasta 12 horas, aunque en la práctica podrían terminarse antes según uso y carga del sistema.
\end{bulletlist}

Por otra parte el lenguaje de programación usado ha sido Python, un lenguaje que es ampliamente utilizado por científicos de datos. En las últimas décadas Python se ha enriquecido con numerosas librerías relacionadas con técnicas de ML que facilitan el uso de las mismas. En concreto para este proyecto se ha utilizado la versión 3.12.11 de Python.

En cuanto a las librerías de Python usadas para la implementación, se presentan a continuación:

\begin{bulletlist}
\item NumPy: es una librería que ofrece la posibilidad de crear matrices y vectores multidimensionales y provee además un gran número de operaciones matemáticas de alto nivel.
\item Pandas: es una librería que ofrece la estructura de datos llamada DataFrame que facilita la manipulación y el análisis de datos. Es una extensión de la librería NumPy. Ha sido usada para tratar y transformar los datos.
\item Plantcv: es una librería de Python de código abierto diseñada específicamente para el análisis de imágenes de plantas. Se ha usado para lectura de las imágenes.
\item Tensorflow: es una librería de software de código abierto creada por Google para desarrollar y entrenar modelos de machine learning (ML) y deep learning (DL). Recibe este nombre porque trabaja con tensores, estructuras de datos multidimensionales, como matrices o vectores que fluyen a través de un grafo computacional de operaciones. Se ha usado para crear y entrenar los modelos descritos en este proyecto.
\item Seaborn: esta librería permite la visualización de los datos a través de distintos tipos de gráficas. Ha sido usada para realizar los distintos gráficos como las matrices de confusión para evaluar los modelos.
\item Sklearn: es una librería que contiene un gran número funciones relacionadas con modelos de machine learning (ML) y deep learning (DL). Se ha usado para extraer las principales métricas de los modelos tras su entrenamiento.
\end{bulletlist}


\section{Procedencia y descripción de los datos}\label{sec:sec4.2}

Los datos provienen de la plataforma online de ciencia de datos de Google, Kaggle, que funciona como una mezcla de red social, repositorio de datasets y espacio de competición. En concreto, el conjunto de datos usuado es el llamado "Tomato Leaves Dataset". Según su descripción en la misma plataforma se trata de un conjunto de datos de más de 20.000 imágenes de hojas de tomate con 11 clases, 10 enfermedades y una clase sana. Estas imágenes se han recopilado tanto en entornos de laboratorio como en entornos naturales. \parencite{kaggle}

En concreto se pueden extraer dos directorios que servirán como conjunto de datos para el entrenamiento y conjunto de datos para validación, ambos cuentan con 11 directorios con imágenes dentro. Cada uno de estos subdirectorios representa una de las clases que serán brevemente expuestas a continuación:

\begin{table}[ht] 
\centering
\caption{\headlinecolor{\underline{Clases del conjunto de datos}}}
\label{tab:clases_datos}
\begin{tabular}{|c|c|} \hline
\rowcolor{naranja}
\multicolumn{2}{|c|}{\textbf{Conjunto de datos}} \\ \hline
\rowcolor{naranja!30}
Clase & Traducción Clase \\ \hline 
\texttt{Healthy} & Saludable \\ \hline 
\texttt{Bacterial\_spot} & Manchas bacterianas \\ \hline 
\texttt{Early\_blight} & Tizón precoz \\ \hline 
\texttt{Late\_blight} & Tizón tardío \\ \hline 
\texttt{Leaf\_Mold} & Hojas con moho \\ \hline 
\texttt{Powdery\_mildew} & Moho polvoriento \\ \hline 
\texttt{Septoria\_leaf\_spot} & Hojas manchadas de septoriosis \\ \hline 
\texttt{Spidermite\_Two-spotted\_spider\_mite} & Picadura de araña roja de dos manchas \\ \hline 
\texttt{Target\_Spot} & Punto blanco \\ \hline 
\texttt{Tomato\_mosaic\_virus} & Virus mosaico \\ \hline 
\texttt{Tomato\_Yellow\_Leaf\_Curl\_Virus} & Virus de la hoja amarilla \\ \hline
\end{tabular}
\end{table}


\section{Preprocesado de los datos}\label{sec:sec4.3}

Al realizar la carga de datos se tiene un directorio con dos subdirectorios, cada uno de ellos representará un conjunto de datos, uno de datos de entrenamiento (\textit{train}) y otro de validación (\textit{valid}). Cada uno de estos directorios contienen a su vez 11 directorios, representando cada una de las clases. 

A continuación, se llevan a cabo las primeras tareas de exploración de las imágenes.

En primer lugar, seleccionando el directorio que contiene los datos de entrenamiento se realiza una función para obtener el número de imágenes existentes por cada tipo de tamaño. Gracias a esta función se sabe que se cuentan con imágenes de variables tamaños, concretamente existen 760 tipos de tamaños distintos. En la siguiente tabla se exponen los 5 tipos de tamaño que más se repiten:

\begin{table}[ht] 
\centering
\caption{\headlinecolor{\underline{Los 5 tamaños de imágenes más comunes}}}
\label{tab:tamanos_imagenes}
\begin{tabular}{|c|c|} \hline
\rowcolor{naranja}
\multicolumn{2}{|c|}{\textbf{Top 5 tamaños de imágenes}} \\ \hline
\rowcolor{naranja!30}
Tamaño (píxeles) & N\textsuperscript{o} imágenes \\ \hline 
256x256 & 18942 \\ \hline 
227x227 & 4120 \\ \hline 
640x640 & 1207 \\ \hline 
533x800 & 151 \\ \hline 
800x600 & 10 \\ \hline
\end{tabular}
\end{table}

De cara a construir un modelo todas las imágenes tienen que tener el mismo tamaño y debida a esta primera toma de contacto se toma la decisión de transformar todas las imágenes a tamaño de 256x256 píxeles. Sin embargo, debido a los modelos usados se termina convirtiendo a imágenes de tamaño 224x224, ya que es un tamañao estándar que entiende cualquier modelo y es el más cercano a los dos tipo de tamños de imágenes más usuales en nuestros datos.

En segundo lugar, se realiza una muestra aleatoria de una imagen por clase, para visualizar el tipo de imágenes que se van a tratar en la Figura \ref{fig:random_per_class}.

\begin{figure}[ht]
\begin{center}
\includegraphics[scale=0.4]{./Images/Preprocesado/Imagen_aleatoria_por_clase.png}
\caption{\headlinecolor{\underline{Imagen aleatoria por clase}}}
\label{fig:random_per_class}
\end{center}
\end{figure}

En la Figura \ref{fig:random_per_class} se puede observar la gran variedad de imágenes que existen, con distintos brillos, distintos fondos o incluso giradas:

\begin{bulletlist}
\item En cuanto al brillo se puede observar que la imagen de \texttt{Target\_Spot} tiene mucho más brillo que la de \texttt{Tomato\_Yellow\_Leaf\_Curl\_Virus}.
\item Con respecto al fondo podemos ver distinciones entre la imagen \texttt{Bacterial\_spot} con un fondo grisáceo plano, la imagen \texttt{Late\_blight} con fondo completamente negro y la foto \texttt{Early\_blight} en la que se ve el resto de la planta de tomate, no solo se ve una hoja.
\item También se puede destacar la diferencia entre las posiciones de las hojas, algunas como \texttt{Bacterial\_spot} tienen el tallo abajo, otras como \texttt{Leaf\_Mold} tienen el tallo arriba y otras como \texttt{Septoria\_leaf\_spot} tienen el tallo horizontal. 
\item Además se tiene un ejemplo de imagen rotada, concretamente la \texttt{Septoria\-\_leaf\_spot}
\end{bulletlist}

Debido a esta gran variedad de imágenes y de enfermedades se llegó a la conclusión de que no tiene mucho sentido usar funciones especiales de la librería PlantCV, ya que esta está muy orientada al fenotipado clásico (área, forma, color, índices), y esas características a veces no capturan la complejidad de patrones de enfermedades, que suelen ser más sutiles y no lineales. La estrategia que se ha seguido es pasar a un pipeline de \textit{deep learning} con imágenes preprocesadas de forma estándar. De esta manera el modelo aprenderá por si mismo las características discriminantes en lugar de imponer un conjunto de \textit{features} manuales.

A continuación se dispuso a formar los conjuntos de datos que usará el modelo. Hasta ahora se tienen datos para el entrenamiento del modelo y para la validación del mismo, sin embargo, no se tienen datos para realizar pruebas sobre el modelo resultante. Por lo tanto, se formará un nuevo conjunto de datos de pruebas a partir del cunjunto de entrenamiento, concretamente, seleccionando un 20\% de sus datos.

Para realizar esta tarea se recorre el directorio de datos de entrenamiento y aleatoriamente se seleccionan imágenes de cada subdirectorio (clase) y se añaden a un nuevo directorio que será el de datos de prueba.

Otra tarea importante en cuanto al procesado de los datos es normalizar los mismos. Para ello, se hace uso de \textit{ImageDatagenerator} de Tensorflow. Se usa esta función porque también sirve para aplicar la técnica de aumento de datos (\textit{data augmentation}) \parencite{research12} en los datos de entrenamiento. Esta técnica consiste en generar datos adicionales a partir de los ya existentes aplicando transformaciones que mantienen la esencia de la información original, pero la presentan de manera distinta. Su objetivo principal es enriquecer el conjunto de entrenamiento para mejorar la capacidad de generalización del modelo resultante.

Es decir, a los tres conjuntos de datos se le apliza normalización que consiste en transformar los pixeles de cada imagen de valores que van de 0 a 255 (escala RGB) a valores entre 0 y 1. Además a las imágenes del conjunto de entrenamiento se le aplica la técnica de \textit{data augmentation} que se ha mencionado anteriormente, concretamente se le aplican las siguientes transformaciones:

\begin{bulletlist}
\item Rota aleatoriamente la imagen hasta más menos 20 grados.
\item Aplica un zoom aleatorio entre 80\% y 120\% del tamaño original.
\item Desplaza la imagen horizontalmente hasta un 20\% de su ancho.
\item Desplaza la imagen verticalmente hasta un 20\% de su alto.
\item Aplica una transformación de cizallamiento (\textit{shear}), como si se deformara la imagen en diagonal.
\item Gira horizontalmente las imágenes aleatoriamente.
\item Cambia el brillo aleatoriamente entre 80\% y 120\%.
\item Cuando una transformación (como una rotación o desplazamiento) deja espacios vacíos en la imagen, estos se rellenan con el valor del píxel más cercano.
\end{bulletlist}

Estas transformaciones se aplicarán aleatoriamente cada vez que el generador entrega un lote al modelo durante el entrenamiento. De esta manera, el modelo el modelo nunca verá dos veces exactamente la misma versión de la imagen.

Como ya se ha mencionado se ha usado \textit{ImageDatagenerator} para formar los conjuntos de datos. Concretamente lo que permite es crear generadores, que no son más que objetos que producen datos de forma incremental, lote a lote, en lugar de cargar todo el \textit{dataset} en memoria de golpe. El generador se conecta a un directorio con datos y los va leyendo en lotes de un tamaño determinado, en este caso se ha usado un tamaño de 32, es decir, va formando lotes de 32 imágnes y a estas se les aplica normalización y también las transformaciones de \textit{data augmentation} mencionadas si son datos de entrenamiento.

De esta manera, quedan los siguientes conjuntos de datos:

\begin{bulletlist}
\item Entrenamiento: Tiene 20.686 imágenes utilizadas para entrenar el modelo. Se trata de la mayor parte de los datos, ya que el modelo necesita muchos ejemplos para aprender patrones.
\item Validación: Tiene 6.683 imágenes. Estos datos se utilizan para evaluar el rendimiento del modelo durante el entrenamiento, sin afectar a los parámetros del modelo.
\item Prueba: Hay 5.165 imágenes para la prueba final. Este conjunto de datos se utiliza una vez finalizado el entrenamiento para medir objetivamente el rendimiento del modelo con datos nuevos que nunca se han visto.
\end{bulletlist}

Otro dato importante del conjunto de entrenamiento es el número de imágenes que se tiene por clase, ya que si hubiera muchas más imágenes de una clase que de otras, el modelo podría incluir un sesgo no deseado. El número de imágenes por clase del conjunto de entrenamiento es el siguiente:


\begin{table}[ht] 
\centering
\caption{\headlinecolor{\underline{Número de imágenes por clases en el conjunto de entrenamiento}}}
\label{tab:num_imagen_por_clase_entrenamiento}
\begin{tabular}{|c|c|} \hline
\rowcolor{naranja}
\multicolumn{2}{|c|}{\textbf{Número de imágenes por clases}} \\ \hline
\rowcolor{naranja!30}
Clase & N\textsuperscript{o} imágenes \\ \hline 
\texttt{Bacterial\_spot} & 2261 \\ \hline 
\texttt{Early\_blight} & 1964 \\ \hline 
\texttt{Late\_blight} & 2491 \\ \hline 
\texttt{Leaf\_Mold} & 2204 \\ \hline 
\texttt{Septoria\_leaf\_spot} & 2306 \\ \hline
\texttt{Spidermite\_Two-spotted\_spider\_mite} & 1398 \\ \hline
\texttt{Target\_Spot} & 1462 \\ \hline
\texttt{Tomato\_Yellow\_Leaf\_Curl\_Virus} & 1632 \\ \hline
\texttt{Tomato\_mosaic\_virus} & 1723 \\ \hline
\texttt{healthy} & 2441 \\ \hline
\texttt{powdery\_mildew} & 804 \\ \hline
\end{tabular}
\end{table}

La mayoría de clases tienen entre 2000 y 2400 imágenes, lo cual es aceptable. Sin embargo, existen clases claramente minoritarias, como \texttt{Spidermite\_Two-spotted\-\_spider\_mite} y \texttt{Target\_Spot} con menos de 1500 imágenes, o  \texttt{powdery\_mildew} con menos de 1000 imágenes. Esto puede causar que el modelo aprenda mejor las clases con más ejemplos y tienda a confundirse en las minoritarias porque tiene menos exposición a ellas.

Para evitar el posible problema de sesgo en el modelo se aplicará una técnica de balanceo al conjunto de entrenamiento. Específicamente se usará ponderación de clases con un diccionario \texttt{class\_weight} que se puede añadir al modelo en forma de parámetro. Esto dará más peso a los errores en clases minoritarias, para que el modelo no las ignore.

Los pesos de las clases se han calculado usando la función \texttt{compute\_class\_weight} que hace que las clases con menor muestras tengan mayor peso y las que tengan más muestras tengan menor peso. Haciendo, en promedio, que todas las clases tengan la misma importancia. Los pesos aplicados son los que se muestran a continuación:

\begin{table}[ht] 
\centering
\caption{\headlinecolor{\underline{Pesos asignados a cada clase en el entrenamiento del modelo}}}
\label{tab:pesos_por_clase_entrenamiento}
\begin{tabular}{|c|c|c|} \hline
\rowcolor{naranja}
\multicolumn{3}{|c|}{\textbf{Pesos por clases}} \\ \hline
\rowcolor{naranja!30}
Clase & N\textsuperscript{o} imágenes & Peso \\ \hline 
\texttt{Bacterial\_spot} & 2261 & 0.8317 \\ \hline 
\texttt{Early\_blight} & 1964 & 0.9575 \\ \hline 
\texttt{Late\_blight} & 2491 & 0.7549 \\ \hline 
\texttt{Leaf\_Mold} & 2204 & 0.8532 \\ \hline 
\texttt{Septoria\_leaf\_spot} & 2306 & 0.8155 \\ \hline
\texttt{Spidermite\_Two-spotted\_spider\_mite} & 1398 & 1.3451 \\ \hline
\texttt{Target\_Spot} & 1462 & 1.2862 \\ \hline
\texttt{Tomato\_Yellow\_Leaf\_Curl\_Virus} & 1632 & 1.1522 \\ \hline
\texttt{Tomato\_mosaic\_virus} & 1723 & 1.0914 \\ \hline
\texttt{healthy} & 2441 & 0.7703 \\ \hline
\texttt{powdery\_mildew} & 804 & 2.3389 \\ \hline
\end{tabular}
\end{table}


\section{Modelado}\label{sec:sec4.4}

En esta sección se describe la estrategia de modelado empleada, común a todos los experimentos realizados posteriormente. El enfoque adoptado se basa en el uso de redes neuronales convolucionales (CNN), dado que constituyen la arquitectura de referencia en tareas de clasificación de imágenes al ser capaces de extraer de manera automática y jerárquica características relevantes de los datos visuales.

Considerando las limitaciones de recursos computacionales disponibles, se optó por la técnica de \textit{transfer learning}. Esta metodología permite aprovechar modelos previamente entrenados sobre grandes bases de datos, de modo que las capas iniciales ya contienen representaciones generales de las imágenes. Posteriormente, dichas representaciones se ajustan a la tarea específica de clasificación de enfermedades en hojas de tomate.

Concretamente se han usado los modelos \textit{MobileNetV2}, \textit{NASNetMobile} y \textit{EfficientNetB0} preentrenados con el dataset de \textit{ImageNet}. De esta manera se aprovechan los pesos previamente aprendidos, que contienen representaciones visuales generales como bordes, texturas y formas, para inicializar la red. Posteriormente, esta red se adapta a la clasificación de enfermedades en hojas de tomate. 

Además, cabe mencionar que para algunos modelos se ha usado la técnica de \textit{fine tuning} que va un poco más allá que \textit{transfer learning}, ya que tras entrenar un modelo con una gran base de datos se descongelan algunas de sus capas para que el modelo final se adapte más a la tarea específica que en este caso es la clasificación de enfermedades en hojas de tomate.

Al haber usado la técnica de \textit{fine tuning} tiene sentido añadir las características principales de las arquitecturas de redes neuronales usadas. Esta tarea se lleva a cabo en la sección \ref{sec:sec4.5}.
Con el objetivo de gestionar mejor los recursos disponibles, se implementaron disitntos callbacks:

\begin{bulletlist}
\item \textit{HistorySaver}: permite guardar el historial de métricas en un archivo externo. Esto resulta especialmente útil en entornos con recursos limitados o sesiones interrumpibles (como Google Colab), ya que garantiza que la información del entrenamiento no se pierda. Concretamente se ha usado para guardar las métricas de precisión y pérdida de entrenamiento y validación.
\item \textit{ModelCheckpoint}: encargado de almacenar en disco el modelo con mejor desempeño en validación, según la métrica  de precisión (\texttt{val\_accuracy}). De este modo, se asegura la conservación de la mejor versión del modelo entrenado, evitando depender únicamente de los pesos finales.
\end{bulletlist}

Estos callbacks se han usado en todos los entrenamientos. Sin embargo, en algunos entrenamientos se han usado además otros dos callbacks:

\begin{bulletlist}
\item \textit{EarlyStopping}: detiene el entrenamiento de forma anticipada si la pérdida de validación no mejora durante un número determinado de épocas consecutivas. Esto evita sobreentrenamiento y reduce el tiempo de cómputo innecesario.
\item \textit{ReduceLROnPlateau}: ajusta de manera dinámica la tasa de aprendizaje cuando la pérdida de validación alcanza una meseta. Gracias a esta reducción progresiva, el modelo puede seguir afinando sus parámetros con pasos cada vez más pequeños, lo que mejora la convergencia.
\end{bulletlist}

En conjunto, estos callbacks permitieron no solo optimizar el uso de los recursos computacionales disponibles, sino también obtener modelos más robustos y con mejor capacidad de generalización.

Por último cabe destacar que para que los modelos puedan ser reconstruidos por cualquiera con acceso a los mismos datos de los que se parte es necesario fijar a un valor fijo disintas semillas aleatorias:

\begin{bulletlist}
\item Para la inicialización de los pesos en el modelo se usa \texttt{tf.random.set\_seed(42)}.
\item Para la generación de los conjuntos de datos con \textit{ImageDatagenerator} se usa el parámetro \texttt{seed} con valor 42.
\item Para el barajado de los datos del conjunto de entrenmaiento \texttt{random.seed(42)} y \texttt{np.random.seed(42)}.
\end{bulletlist}


\subsection{Modelo \textit{MobileNetV2} 1}\label{sec:sec4.4.1}

Este modelo usa de capa base \textit{MobileNetV2} preentrenada con los pesos de \textit{ImageNet}. Posteriormente se congelan las capas del modelo base para evitar que los pesos se modifiquen durante el entrenamiento y se use correctamente la técnica de \textit{transfer learning}. A continuación se añade una capa de \textit{GlobalAveragePooling} y una capa densa con 11 neuronas y función de activación \textit{softmax}, ya que nos enfrentamos a una clasificación de 11 clases.
Esta información es la que se muestra en la Figura \ref{fig:mobilenetv2_1}.

\begin{figure}[ht]
\begin{center}
\includegraphics[scale=0.7]{./Images/Modelos/summary_mobilnetv2_1.png}
\caption{\headlinecolor{\underline{Resumen del modelo \textit{MobileNetV2} 1}}}
\label{fig:mobilenetv2_1}
\end{center}
\end{figure}

Finalmente para la compilación se usa el optimizador Adam, la función de pérdida de entropía cruzada categórica y se escoge la métrica de precisión para evaluar el rendimiento. 

Se usa la función de pérdida de entropía cruzada categórica porque al crear los conjuntos de datos con ImageDatagenerator como se indica en la sessción \ref{sec:sec4.3} se usa el parámetro class mode con valor categorical que convierte las etiquetas a vectores one-hot encoded.


\subsection{Modelo \textit{MobileNetV2} 2}\label{sec:sec4.4.2}

Este modelo usa de capa base \textit{MobileNetV2} preentrenada con los pesos de \textit{ImageNet}. Posteriormente se congelan las capas del modelo base para evitar que los pesos se modifiquen durante el entrenamiento, pero en esta ocasión se descongelan algunas capas, usando la técnica de \textit{fine tuning}. Concretamente, se descongela a partir de la capa 100 con el objetivo de que el modelo se adapte mejor a nuestro caso concreto como se explica en el Cuadro \ref{tab:arquitectura_MobileNetV2}. A continuación se añade una capa de \textit{GlobalAveragePooling} y una capa densa con 11 neuronas y función de activación \textit{softmax}, ya que nos enfrentamos a una clasificación de 11 clases. Esta información es la que se muestra en la Figura \ref{fig:mobilenetv2_2}.

\begin{figure}[ht]
\begin{center}
\includegraphics[scale=0.7]{./Images/Modelos/summary_mobilnetv2_2.png}
\caption{\headlinecolor{\underline{Resumen del modelo \textit{MobileNetV2} 2}}}
\label{fig:mobilenetv2_2}
\end{center}
\end{figure}

Finalmente para la compilación se usa el optimizador Adam, la función de pérdida de entropía cruzada categórica y se escoge la métrica de precisión para evaluar el rendimiento. Sin embargo, no se usa un \textit{learning rate} por defecto (\texttt{1e-3}), se usa un \textit{learning rate} menor (\texttt{1e-5}).

Se usa la función de pérdida de entropía cruzada categórica porque al crear los conjuntos de datos con ImageDatagenerator como se indica en la Sección \ref{sec:sec4.3} se usa el parámetro \textit{class mode} con valor categorical que convierte las etiquetas a vectores \textit{one-hot encoded}.


\subsection{Modelo \textit{MobileNetV2} 3}\label{sec:sec4.4.3}

Este modelo usa de capa base \textit{MobileNetV2} preentrenada con los pesos de \textit{ImageNet}. Posteriormente se congelan las capas del modelo base para evitar que los pesos se modifiquen durante el entrenamiento, pero en esta ocasión se descongelan algunas capas, usando la técnica de \textit{fine tuning}. Concretamente, se descongela a partir de la capa 80 con el objetivo de que el modelo se adapte mejor a nuestro caso concreto como se explica en el Cuadro \ref{tab:arquitectura_MobileNetV2}. A continuación se añade una capa de \textit{GlobalAveragePooling} y una capa densa con 11 neuronas y función de activación \textit{softmax}, ya que nos enfrentamos a una clasificación de 11 clases. Esta información es la que se muestra en la Figura \ref{fig:mobilenetv2_3}.

\begin{figure}[ht]
\begin{center}
\includegraphics[scale=0.7]{./Images/Modelos/summary_mobilnetv2_2.png}
\caption{\headlinecolor{\underline{Resumen del modelo \textit{MobileNetV2} 3}}}
\label{fig:mobilenetv2_3}
\end{center}
\end{figure}

Finalmente para la compilación se usa el optimizador Adam, la función de pérdida de entropía cruzada categórica y se escoge la métrica de precisión para evaluar el rendimiento. Sin embargo, no se usa un \textit{learning rate} por defecto (\texttt{1e-3}), se usa un \textit{learning rate} menor (\texttt{5e-6}).

Se usa la función de pérdida de entropía cruzada categórica porque al crear los conjuntos de datos con ImageDatagenerator como se indica en la Sección \ref{sec:sec4.3} se usa el parámetro \textit{class mode} con valor categorical que convierte las etiquetas a vectores \textit{one-hot encoded}.


\subsection{Modelo \textit{EfficientNetB0} 1}\label{sec:sec4.4.4}

Este modelo usa de capa base \textit{EfficientNetB0} preentrenada con los pesos de \textit{ImageNet}. Posteriormente se congelan las capas del modelo base para evitar que los pesos se modifiquen durante el entrenamiento y se use correctamente la técnica de \textit{transfer learning}. A continuación se añade una capa de \textit{GlobalAveragePooling} y una capa densa con 11 neuronas y función de activación \textit{softmax}, ya que nos enfrentamos a una clasificación de 11 clases.
Esta información es la que se muestra en la Figura \ref{fig:efficientnetb0_1}.

\begin{figure}[ht]
\begin{center}
\includegraphics[scale=0.7]{./Images/Modelos/summary_efficientnetb0_1.png}
\caption{\headlinecolor{\underline{Resumen del modelo \textit{EfficientNetB0} 1}}}
\label{fig:efficientnetb0_1}
\end{center}
\end{figure}

Finalmente para la compilación se usa el optimizador Adam, la función de pérdida de entropía cruzada categórica y se escoge la métrica de precisión para evaluar el rendimiento. 

Se usa la función de pérdida de entropía cruzada categórica porque al crear los conjuntos de datos con ImageDatagenerator como se indica en la Sección \ref{sec:sec4.3} se usa el parámetro \textit{class mode} con valor categorical que convierte las etiquetas a vectores \textit{one-hot encoded}.



\subsection{Modelo \textit{EfficientNetB0} 2}\label{sec:sec4.4.5}

Este modelo usa de capa base \textit{EfficientNetB0} preentrenada con los pesos de \textit{ImageNet}. Posteriormente se congelan las capas del modelo base para evitar que los pesos se modifiquen durante el entrenamiento, pero en esta ocasión se descongelan algunas capas, usando la técnica de \textit{fine tuning}. Concretamente, se descongela a partir de la capa 100 con el objetivo de que el modelo se adapte mejor a nuestro caso concreto como se explica en el Cuadro \ref{tab:arquitectura_EfficientNetB0}. A continuación se añade una capa de \textit{GlobalAveragePooling} y una capa densa con 11 neuronas y función de activación \textit{softmax}, ya que nos enfrentamos a una clasificación de 11 clases. Esta información es la que se muestra en la Figura \ref{fig:efficientnetb0_2}.

\begin{figure}[ht]
\begin{center}
\includegraphics[scale=0.7]{./Images/Modelos/summary_efficientnetb0_2.png}
\caption{\headlinecolor{\underline{Resumen del modelo \textit{EfficientNetB0} 2}}}
\label{fig:efficientnetb0_2}
\end{center}
\end{figure}

Finalmente para la compilación se usa el optimizador Adam, la función de pérdida de entropía cruzada categórica y se escoge la métrica de precisión para evaluar el rendimiento. Sin embargo, no se usa un \textit{learning rate} por defecto (\texttt{1e-3}), se usa un \textit{learning rate} menor (\texttt{1e-5}).

Se usa la función de pérdida de entropía cruzada categórica porque al crear los conjuntos de datos con ImageDatagenerator como se indica en la Sección \ref{sec:sec4.3} se usa el parámetro \textit{class mode} con valor categorical que convierte las etiquetas a vectores \textit{one-hot encoded}.



\subsection{Modelo \textit{EfficientNetB0} 3}\label{sec:sec4.4.6}

Este modelo usa de capa base \textit{EfficientNetB0} preentrenada con los pesos de \textit{ImageNet}. Posteriormente se congelan las capas del modelo base para evitar que los pesos se modifiquen durante el entrenamiento, pero en esta ocasión se descongelan algunas capas, usando la técnica de \textit{fine tuning}. Concretamente, se descongela a partir de la capa 100 con el objetivo de que el modelo se adapte mejor a nuestro caso concreto como se explica en el Cuadro \ref{tab:arquitectura_EfficientNetB0}. A continuación se añade una capa de \textit{GlobalAveragePooling} y una capa densa con 11 neuronas y función de activación \textit{softmax}, ya que nos enfrentamos a una clasificación de 11 clases. Esta información es la que se muestra en la Figura \ref{fig:efficientnetb0_2}.

\begin{figure}[ht]
\begin{center}
\includegraphics[scale=0.7]{./Images/Modelos/summary_efficientnetb0_3.png}
\caption{\headlinecolor{\underline{Resumen del modelo \textit{EfficientNetB0} 3}}}
\label{fig:efficientnetb0_3}
\end{center}
\end{figure}

Finalmente para la compilación se usa el optimizador Adam, la función de pérdida de entropía cruzada categórica y se escoge la métrica de precisión para evaluar el rendimiento. Sin embargo, no se usa un \textit{learning rate} por defecto (\texttt{1e-3}), se usa un \textit{learning rate} menor (\texttt{3e-6}).

Se usa la función de pérdida de entropía cruzada categórica porque al crear los conjuntos de datos con ImageDatagenerator como se indica en la Sección \ref{sec:sec4.3} se usa el parámetro \textit{class mode} con valor categorical que convierte las etiquetas a vectores \textit{one-hot encoded}.



\subsection{Modelo \textit{NASNetMobile} 1}\label{sec:sec4.4.7}

Este modelo usa de capa base \textit{NASNetMobile} preentrenada con los pesos de ImageNet. Posteriormente se congelan las capas del modelo base para evitar que los pesos se modifiquen durante el entrenamiento y se use correctamente la técnica de \textit{transfer learning}. A continuación se añade una capa de \textit{GlobalAveragePooling} y una capa densa con 11 neuronas y función de activación \textit{softmax}, ya que nos enfrentamos a una clasificación de 11 clases. Esta información es la que se muestra en la Figura \ref{fig:nasnetmobile_1}.

\begin{figure}[ht]
\begin{center}
\includegraphics[scale=0.7]{./Images/Modelos/summary_nasnetmobile_1.png}
\caption{\headlinecolor{\underline{Resumen del modelo \textit{NASNetMobile} 1}}}
\label{fig:nasnetmobile_1}
\end{center}
\end{figure}

Finalmente para la compilación se usa el optimizador Adam, la función de pérdida de entropía cruzada categórica y se escoge la métrica de precisión para evaluar el rendimiento. 



\subsection{Modelo \textit{NASNetMobile} 2}\label{sec:sec4.4.8}

Este modelo usa de capa base \textit{NASNetMobile} preentrenada con los pesos de \textit{ImageNet}. Posteriormente se congelan las capas del modelo base para evitar que los pesos se modifiquen durante el entrenamiento, pero en esta ocasión se descongelan algunas capas, usando la técnica de \textit{fine tuning}. Concretamente, se descongela a partir de la capa 60 con el objetivo de que el modelo se adapte mejor a nuestro caso concreto como se explica en el Cuadro \ref{tab:arquitectura_NASNetMobile}. A continuación se añade una capa de \textit{GlobalAveragePooling} y una capa densa con 11 neuronas y función de activación \textit{softmax}, ya que nos enfrentamos a una clasificación de 11 clases. Esta información es la que se muestra en la Figura \ref{fig:nasnetmobile_2}.

\begin{figure}[ht]
\begin{center}
\includegraphics[scale=0.7]{./Images/Modelos/summary_nasnetmobile_2.png}
\caption{\headlinecolor{\underline{Resumen del modelo \textit{NASNetMobile} 2}}}
\label{fig:nasnetmobile_2}
\end{center}
\end{figure}

Finalmente para la compilación se usa el optimizador Adam, la función de pérdida de entropía cruzada categórica y se escoge la métrica de precisión para evaluar el rendimiento. Sin embargo, no se usa un \textit{learning rate} por defecto (\texttt{1e-3}), se usa un \textit{learning rate} menor (\texttt{5e-6}).

Se usa la función de pérdida de entropía cruzada categórica porque al crear los conjuntos de datos con ImageDatagenerator como se indica en la Sección \ref{sec:sec4.3} se usa el parámetro \textit{class mode} con valor categorical que convierte las etiquetas a vectores \textit{one-hot encoded}.



\section{Arquitecturas}\label{sec:sec4.5}

En esta sección se explicarán las principales características de las arquitecturas empleadas para construir los modelos de la sección \ref{sec:sec4.4}

\subsection{\textit{MobileNetV2}}\label{sec:sec4.5.1}

\textit{MobileNetV2} es una arquitectura de CNN muy efieciente diseñada para aplicaciones de visión integrada en móviles. Ha sido desarrollada por Google para mejorar a su antedecesor \textit{MobileNetV1} proporcionando una mejor precisión y reduciendo la capacidad de cómputo. Sus características claves son las siguientes \parencite{MobilenetV2Architecture}:

\begin{enumerate}[label=\bfseries\headlinecolor\arabic*.]
\item Bloques residuales invertidos (\textit{Inverted Residuals}): Primero se expande el número de canales mediante una convolución 1x1 (capa de expansión). Luego se aplica una convolución \textit{"depthwise"} (convolución separada por canal) para realizar el filtrado espacial. Finalmente, se reduce de nuevo el número de canales a la dimensión deseada mediante otra 1x1 (capa de proyección). Esta estructura permite conservar dimensiones de entrada/salida mientras se reduce el coste computacional.
\item Convoluciones separables en profundidad (\textit{Depthwise Separable Convolutions}): Al igual que su antedecesor, utiliza convoluciones separables, es decir, se descompone una convolución estándar en dos operaciones; primero una convolución \textit{"depthwise"} por cada canal, luego una \textit{"pointwise”} 1x1 que mezcla los canales. Esto reduce considerablemente el número de parámetros y de cálculos, lo que mejora la eficiencia.
\item Embotellamientos lineales (\textit{Linear Bottlenecks}): Entre capas se insertan \textit{bottlenecks} lineales, que ayudan a que la "variedad" (\textit{manifold}) de los datos de entrada no se comprima excesivamente. Esta práctica contribuye a conservar mayor información, lo cual mejora la precisión del modelo.
\item Función de activación ReLU6: emplea la función de activación ReLU6, una versión modificada de la ReLU clásica que limita los valores de activación al rango [0, 6]. Esto favorece la cuantización (es decir, el uso eficiente de formatos de menor precisión) en dispositivos móviles manteniendo un buen equilibrio entre eficiencia y precisión.
\end{enumerate}

A continuación se realiza una división funcional de la red para entender qué aprenden las capas \parencite{}: %%https://uk.mathworks.com/help/deeplearning/ref/mobilenetv2.html

\begin{bulletlist}
\item Zona temprana: Comprende los primeros bloques, aproximadamente hasta la capa 20. En esta etapa, la red aprende características muy básicas, como bordes, texturas simples, colores, orientación y gradientes. Se trata de filtros universales que casi cualquier red visual aprende.
\item Zona intermedia: Incluye los bloques medios, aproximadamente desde la capa 20 hasta la 80. Aquí se capturan características más complejas, como combinaciones de formas, patrones locales, partes de objetos y texturas específicas (por ejemplo, de hojas o plantas). En este punto, la red comienza a adaptarse a dominios más específicos.
\item Zona tardía o cabeza del modelo: corresponde a los últimos bloques, desde la capa 80 hasta la final. Esta zona extrae características de muy alto nivel, relacionadas con la semántica del dominio, las relaciones globales dentro de la imagen y la adaptación al problema concreto que se desea resolver.
\end{bulletlist}



\subsection{\textit{EfficientNetB0}}\label{sec:sec4.5.2}

\textit{EfficientNet} es una familia de redes neuronales convolucionales (CNN) cuyo objetivo es lograr un alto rendimiento con menos recursos computacionales en comparación con arquitecturas anteriores. Sus características claves son las siguientes \parencite{Efficientnet}:

\begin{enumerate}[label=\bfseries\headlinecolor\arabic*.]
\item Tallo (\textit{Stem}): Comienza con una etapa de stem que consiste en una convolución estándar seguida de normalización por lotes (\textit{Batch Normalization}) y activación ReLU6.
\item Cuerpo (\textit{Body}): se compone de una serie de bloques MBConv con diferentes configuraciones (expansión de canales, tamaños de kernel, stride, mecanismo \textit{"squeeze-and-excitation"}).
\item Head (\textit{Cabeza}): Incluye un bloque convolutional final para clasificación.
\item Bloques internos: Al igual que \textit{MobileNetV2} usa convoluciones separables en profundidad (\textit{Depthwise Separable Convolutions}) y bloques residuales invertidos (\textit{Inverted Residuals}). Además añade bloques \textit{Squeeze-and-Excitation} para calibrar la importancia de los canales.
\end{enumerate}

A continuación se realiza una división funcional de la red para entender qué aprenden las capas \parencite{}:

\begin{bulletlist}
\item Zona temprana: Desde la entrada (\textit{stem}) hasta los primeros bloques \textit{MBConv}, aproximadamente hasta la capa 20. En esta etapa, la red aprende características muy básicas, como bordes, texturas simples, colores, orientación y gradientes. Se trata de filtros universales que casi cualquier red visual aprende.
\item Zona intermedia: Bloques medios de \textit{MBConv} con mayor expansión y mayor profundidad, aproximadamente de la capa 20 a la 100. Aquí se capturan características más complejas, como combinaciones de formas, patrones locales, partes de objetos y texturas específicas (por ejemplo, de hojas o plantas). En este punto, la red comienza a adaptarse a dominios más específicos.
\item Zona tardía: corresponde a los últimos bloques, desde la capa 100 hasta la final. Esta zona extrae características de muy alto nivel, relacionadas con la semántica del dominio, las relaciones globales dentro de la imagen y la adaptación al problema concreto que se desea resolver.
\end{bulletlist}


\subsection{\textit{NASNetMobile}}\label{sec:sec4.5.3}

\textit{NASNetMobile} es una versión optimizada para dispositivos móviles de la arquitectura \textit{NASNet} (\textit{Neural Architecture Search Network}) desarrollada por \textit{Google Brain}. Su diseño se basa en la búsqueda automática de arquitectura NAS para encontrar estructuras de red eficientes para tareas de clasificación de imágenes de gran escala. Sus características claves son las siguientes \parencite{}:

\begin{enumerate}[label=\bfseries\headlinecolor\arabic*.] 
\item La entrada típica es de 224x224 píxeles en el caso móvil.
\item Su diseño emplea dos tipos de células (\textit{cells}): las llamadas células normales (\textit{normal cells}) y células de reducción (\textit{reduction cells}).
\item Las \textit{normal cells} producen mapas de características que conservan la resolución espacial (altura x anchura).
\item Las \textit{reduction cells} reducen la resolución espacial (por ejemplo, con factor 2 en altura y anchura) para avanzar hacia características más globales.
\end{enumerate}

A continuación se realiza una división funcional de la red para entender qué aprenden las capas \parencite{}:

\begin{bulletlist}
\item Zona temprana: Comprende los primeros bloques, aproximadamente hasta la capa 20. En esta etapa, la red aprende características muy básicas, como bordes, texturas simples, colores, orientación y gradientes. Se trata de filtros universales que casi cualquier red visual aprende.
\item Zona intermedia: Incluye los bloques medios, aproximadamente desde la capa 20 hasta la 60. Aquí se capturan características más complejas, como combinaciones de formas, patrones locales, partes de objetos y texturas específicas (por ejemplo, de hojas o plantas). En este punto, la red comienza a adaptarse a dominios más específicos.
\item Zona tardía o cabeza del modelo: corresponde a los últimos bloques, desde la capa 60 hasta la final. Esta zona extrae características de muy alto nivel, relacionadas con la semántica del dominio, las relaciones globales dentro de la imagen y la adaptación al problema concreto que se desea resolver.
\end{bulletlist}


\chapter{Evaluación y resultados}\label{cap:cap5}

En este capítulo se presentan los resultados obtenidos al usar los datos preprocesados explicado en la sección \ref{sec:sec4.3} a los modelos descritos en la sección \ref{sec:sec4.4}. Para comenzar, se considera necesario realizar una breve explicación sobre las métricas que ayudan a realizar esta tarea (\ref{sec:sec5.1}). Finalmente se realizará una comparación entre los modelos con mejor rendimiento en la Sección \ref{sec:sec5.10}.


\section{Métricas}\label{sec:sec5.1}

El rendimiento de los modelos propuestos se evalúan utilizando diferentes métricas como, la precisión, la sensibilidad, \textit{F1-score} y soporte. Las métricas de evaluación se establecen utilizando fórmulas que usan TP, TN, FP y FN, que denotan verdaderos positivos, verdaderos negativos, falsos positivos y falsos negativos, respectivamente. De esta manera se definen las siguientes métricas:

\begin{bulletlist}
  \item Precisión (\textit{Precision}): Mide la proporción de verdaderos positivos sobre el total de predicciones positivas.
    
    \begin{equation*}
      \text{Precisión} = \frac{TP}{TP + FP}
    \end{equation*}
    
    De cara a la interpretación de lo que nos informa es de todas las imágenes que el modelo predijo como una clase, cuántas realmente pertenecen a esa clase.

  \item Sensibilidad (\textit{Recall}): mide la proporción de verdaderos positivos detectados.

    \begin{equation*}
      \text{Recall} = \frac{TP}{TP + FN}
    \end{equation*}

    Nos informa de todas las imágenes que son realmente de una clase, cuántas fueron detectadas correctamente.
  
  \item \textit{F1-Score}: Es el promedio armónico entre precisión y sensibilidad.
    
    \begin{equation*}
      \text{F1-Score} \frac{\text{Precisión} \times \text{Recall}}{\text{Precisión} + \text{Recall}}
    \end{equation*}


\end{bulletlist}

Otra métrica útil para la evaluación de los modelos usada es la matriz de confusión. Se trata de una tabla que compara las predicciones del modelo con las etiquetas reales, mostrando en su diagonal principal el número de aciertos (instancias correctamente clasificadas) y fuera de la diagonal los errores (instancias que el modelo confundió con otra clase). Cada fila representa las predicciones reales y cada columna las predicciones del modelo, lo que permite identificar con precisión qué clases son más propensas a ser confundidas entre sí.



\section{Modelo \textit{MobileNetV2} 1}\label{sec:sec5.2}

En esta sección se van a mostrar los resultados del entrenamiento del modelo descrito en la subsección \ref{sec:sec4.4.1}. A este modelo se han aplicado varios entrenamientos distintos explicados en las subsecciones \ref{sec:sec5.1.1}, \ref{sec:sec5.1.2} y \ref{sec:sec5.1.3}.


\subsection{Entrenamiento 1}\label{sec:sec5.2.1}

Este entrenamiento se ha realizado usando los pesos de clases descritos en el Cuadro \ref{tab:pesos_por_clase_entrenamiento}. Solo se ha sometido a 10 épocas y se han usado los callbacks de \textit{ModelCheckpoint} e \textit{HistorySaver}, ya que el objetivo de este entrenamiento es observar la tendencia inicial para poderlo comparar con el entrenamiento de la sección \ref{sec:sec5.1.2} y poder determinar qué tanto puede influir los pesos en las clases.

Para mostrar los resultados de este entrenamiento se va a comenzar mostrando las métricas recogidas en el fichero generado por el callback \textit{HistorySaver} en la Figura \ref{fig:metrics_historic_mobilenetv2_1_1}.

\begin{figure}[ht]
\begin{center}
\includegraphics[scale=0.5]{./Images/Resultados/Histórico Métricas/mobilenetv2_1_1.png}
\caption{\headlinecolor{\underline{Histórico de métricas del entrenamiento 1 del modelo \textit{MobileNetV2} 1}}}
\label{fig:metrics_historic_mobilenetv2_1_1}
\end{center}
\end{figure}

En esta figura se puede observar el comportamiento de las métricas de precisión y pérdida de entrenamiento y validación a lo largo de las épocas a las que se ha sometido el modelo descrito en la subsección \ref{sec:sec4.4.1} en el entrenamiento. A continuación se describe lo que se puede interpretar:  

\begin{bulletlist}
\item Precisión de entrenamiento: Comienza alrededor de 0.6 y sube rápidamente a un valor cercano a 0.8, mostrando que el modelo aprende bien de los datos de entrenamiento.
\item Pérdida de entrenamiento: Disminuye de un valor cercano a 1 hasta cerca de 0.6, lo que confirma que el modelo está optimizando correctamente con la función de pérdida elegida para los datos de entrenamiento.
\item Precisión de validación: Aumenta de un valor cercano a 0.7 a un valor cercano a 0.8, lo que muestra que el modelo aprende bien de datos distintos a los de entrenamiento.
\item Pérdida de validación: Disminuye de un valor cercano a 0.8 a un valor cercano a 0.5, confirmando que la función de pérdida optimiza bien el modelo.
\end{bulletlist}

En resumen, ambas métricas de precisión tienden a subir acercándose a 1 y las métricas de pérdidas a bajar acercándose a 0. Ambas tendencias se realizan de manera progresiva sin señales de \textit{overfitting}. 

La precisión de entrenamiento se estabiliza alrededor de 0.8, con incrementos pequeños al final. Esto puede indicar que el modelo se está acercando a su límite de capacidad con la configuración actual. Para exprimir aún más este modelo con intención de mejorarlo se podría aumentar el número de épocas con el callback \textit{EarlyStopping} para que pare el entrenamiento en caso de que no mejore y el callback de \textit{ReduceLROnPlateau} para que reduzca el \textit{learning rate} en el entrenamiento y haga que no se estanquen sus valores. 

Para poder comparar este entrenamiento con el de la subsección \ref{sec:sec5.1.2} es necesario realizar un análisis de métricas por clases. Para ello se van a mostrar las métricas del método \texttt{classification\_report} de la librería \textit{Sklearn} en la Figura \ref{fig:metrics_summary_mobilenetv2_1_1} y la matriz de confusión en la Figura \ref{fig:confusion_matrix_mobilenetv2_1_1}.

\begin{figure}[ht]
\begin{center}
\includegraphics[scale=0.7]{./Images/Resultados/Métricas resumen/mobilenetv2_1_1.png}
\caption{\headlinecolor{\underline{Resumen de métricas del modelo \textit{MobileNetV2} 1 en el entrenamiento 1}}}
\label{fig:metrics_summary_mobilenetv2_1_1}
\end{center}
\end{figure}

\begin{figure}[ht]
\begin{center}
\includegraphics[scale=0.5]{./Images/Resultados/Matriz de confusión/mobilenetv2_1_1.png}
\caption{\headlinecolor{\underline{Matriz de confusión del modelo \textit{MobileNetV2} 1 en el entrenamiento 1}}}
\label{fig:confusion_matrix_mobilenetv2_1_1}
\end{center}
\end{figure}

Se puede decir que el modelo de clasificación multiclase alcanzó una precisión del 81.4\% y un F1-score medio de 0.81, lo que indica un buen equilibrio entre precisión y capacidad de detección en las once clases evaluadas. Las clases \texttt{Tomato\_Yellow\_Leaf\_Curl\_Virus}, \texttt{Tomato\_mosaic\_virus} y \texttt{healthy} destacan por sus altos valores de F1-score (superiores a 0,89), mostrando que el modelo identifica correctamente los patrones más distintivos de estas categorías. Asimismo, el desempeño equilibrado entre las métricas macro y ponderadas (\textit{weighted}) sugiere que no existe un sesgo notable hacia clases con mayor cantidad de muestras.

Sin embargo, algunas clases como \texttt{Target\_Spot}, \texttt{powdery\_mildew}, \texttt{Early\_blight} y \texttt{Septoria\_leaf\_spot} presentan valores más bajos, reflejando cierta confusión con otras enfermedades de síntomas visualmente similares.

De esta manera la matriz de confusión de la Figura \ref{fig:confusion_matrix_mobilenetv2_1_1} refuerza las conclusiones obtenidas a partir de las métricas de evaluación, mostrando que el modelo clasifica correctamente la mayoría de las muestras en casi todas las clases. Se observa una fuerte concentración de valores en la diagonal principal, especialmente en las clases \texttt{Tomato\_Yellow\_Leaf\_Curl\_Virus}, \texttt{Tomato\_mosaic\_virus} y \texttt{healthy}. Estas clases presentan patrones visuales distintivos que el modelo ha aprendido de forma eficaz. 

En contraste, se aprecian errores recurrentes entre clases con síntomas similares, como \texttt{Early\_blight}, \texttt{Late\_blight} y \texttt{Septoria\_leaf\_spot} donde las confusiones cruzadas son más frecuentes. También se observan predicciones erróneas entre \texttt{Target\_Spot} y otras enfermedades foliares, así como entre \texttt{powdery\_mildew} y \texttt{Leaf\_Mold}, lo que explica los valores más bajos de precisión en estas clases. En conjunto, la matriz de confusión visualiza de manera clara la capacidad del modelo para distinguir la mayoría de las enfermedades, evidenciando a la vez los retos en la diferenciación de patologías con síntomas visualmente parecidos.

Este modelo tiene un comportamiento sólido y equilibrado, generaliza bien y consigue que incluso las clases con menos muestras tengan buenos niveles de recall. 



\subsection{Entrenamiento 2}\label{sec:sec5.2.2}

Este entrenamiento es igual que el anterior, pero sin tener en cuenta los pesos de las clases.

Para mostrar los resultados de este entrenamiento se va a comenzar mostrando las métricas recogidas en el fichero generado por el callback \textit{HistorySaver} en la Figura \ref{fig:metrics_historic_mobilenetv2_1_2}.

\begin{figure}[ht]
\begin{center}
\includegraphics[scale=0.5]{./Images/Resultados/Histórico Métricas/mobilenetv2_1_2.png}
\caption{\headlinecolor{\underline{Histórico de métricas del entrenamiento 2 del modelo \textit{MobileNetV2} 1}}}
\label{fig:metrics_historic_mobilenetv2_1_2}
\end{center}
\end{figure}

En la Figura \ref{fig:metrics_historic_mobilenetv2_1_2} podemos observar el comportamiento de las métricas de precisión y pérdida de entrenamiento y validación a lo largo de las épocas a las que se ha sometido el modelo descrito en la subsección \ref{sec:sec4.4.1} en el entrenamiento. A continuación se describe lo que se puede interpretar:  

\begin{bulletlist}
\item Precisión de entrenamiento: Comienza alrededor de 0.6 y sube rápidamente a un valor cercano a 0.8, mostrando que el modelo aprende bien de los datos de entrenamiento.
\item Pérdida de entrenamiento: Disminuye de un valor cercano a 1 hasta cerca de 0.6, lo que confirma que el modelo está optimizando correctamente con la función de pérdida elegida para los datos de entrenamiento.
\item Precisión de validación: Aumenta de un valor cercano a 0.7 a un valor cercano a 0.8, lo que muestra que el modelo aprende bien de datos distintos a los de entrenamiento.
\item Pérdida de validación: Disminuye de un valor cercano a 0.8 a un valor cercano a 0.5, confirmando que la función de pérdida optimiza bien el modelo.
\end{bulletlist}

En resumen, ambas métricas de precisión tienden a subir acercándose a 1 y las métricas de pérdidas a bajar acercándose a 0. Ambas tendencias se realizan de manera progresiva sin señales de \textit{overfitting}. 

El rendimiento global es prácticamente el mismo que el conseguido con el entrenmaiento de la subsección \ref{sec:sec5.1.1}, por lo que para poder compararlos más detalladamente es necesario realizar un análisis de métricas por clases. Para ello se van a mostrar las métricas del método \texttt{classification\_report} de la librería \textit{Sklearn} en la Figura \ref{fig:metrics_summary_mobilenetv2_1_2} y la matriz de confusión en la Figura \ref{fig:confusion_matrix_mobilenetv2_1_2}.

\begin{figure}[ht]
\begin{center}
\includegraphics[scale=0.7]{./Images/Resultados/Métricas resumen/mobilenetv2_1_2.png}
\caption{\headlinecolor{\underline{Resumen de métricas del modelo \textit{MobileNetV2} 1 en el entrenamiento 2}}}
\label{fig:metrics_summary_mobilenetv2_1_2}
\end{center}
\end{figure}

\begin{figure}[ht]
\begin{center}
\includegraphics[scale=0.5]{./Images/Resultados/Matriz de confusión/mobilenetv2_1_2.png}
\caption{\headlinecolor{\underline{Matriz de confusión del modelo \textit{MobileNetV2} 1 en el entrenamiento 2}}}
\label{fig:confusion_matrix_mobilenetv2_1_2}
\end{center}
\end{figure}

El modelo ha alcanzado una precisión del 81\% y un F1-score medio de 0.81, lo que indica un buen equilibrio entre precisión y capacidad de detección en las once clases evaluadas. Las clases \texttt{Tomato\_Yellow\_Leaf\_Curl\_Virus}, \texttt{Tomato\_mosaic\_virus} y \texttt{healthy} siguen destacando por sus altos valores de F1-score (superiores a 0,89), mostrando que el modelo identifica correctamente los patrones más distintivos de estas clases. Asimismo, el desempeño equilibrado entre las métricas macro y ponderadas (\textit{weighted}) sugiere que no existe un sesgo notable hacia clases con mayor cantidad de muestras.

Sin embargo, algunas clases como \texttt{Target\_Spot}, \texttt{powdery\_mildew}, \texttt{Early\_blight} y \texttt{Septoria\_leaf\_spot} presentan valores más bajos, reflejando cierta confusión con otras enfermedades de síntomas visualmente similares.

La matriz de confusión de La Figura \ref{fig:confusion_matrix_mobilenetv2_1_2} refuerza las conclusiones obtenidas a partir de las métricas de evaluación, mostrando que el modelo clasifica correctamente la mayoría de las muestras en casi todas las clases. Se observa una fuerte concentración de valores en la diagonal principal, especialmente en las clases \texttt{Tomato\_Yellow\_Leaf\_Curl\_Virus}, \texttt{Tomato\_mosaic\_virus} y \texttt{healthy}. Estas clases presentan patrones visuales distintivos que el modelo ha aprendido de forma eficaz. 

En contraste, se aprecian errores recurrentes entre clases con síntomas similares, como \texttt{Early\_blight}, \texttt{Late\_blight} y \texttt{Septoria\_leaf\_spot} donde las confusiones cruzadas son más frecuentes. También se observan predicciones erróneas entre \texttt{Target\_Spot} y otras enfermedades foliares, así como entre \texttt{powdery\_mildew} y \texttt{Leaf\_Mold}, lo que explica los valores más bajos de precisión en estas clases. En conjunto, la matriz de confusión visualiza de manera clara la capacidad del modelo para distinguir la mayoría de las enfermedades, evidenciando a la vez los retos en la diferenciación de patologías con síntomas visualmente parecidos.

En general se obtiene un buen rendimiento, existiendo un buen balance entre precisión y sensibilidad en practicamente todas las clases. Aún así, se pueden observar diferencias en algunas clases con respecto al modelo resultante de la subsección \ref{sec:sec5.1.1}. En concreto se puede determinar una disminución en la métrica de sensibilidad en en clases menos representadas en el \textit{dataset} como las clases \texttt{powdery\_mildew} (de 0.93 a 0.80), \texttt{Septoria\_leaf\_spot} (de 0.71 a 0.61) y \texttt{Spidermite\_Two-spotted\_\-spider\_mite} (de 0.84 a 0.70). Esto demuestra la función de los pesos en las clases, es decir el balanceo de las mismas.

El modelo obtenido en la subsección \ref{sec:sec5.1.1} equilibra mejor los resultados sin sacrificar precisión general, por lo que es la opción más sólida para un sistema de diagnóstico de enfermedades, donde todas las clases tienen importancia clínica. Por tanto  el modelo que se seguirá mejorando en la suubsección subsección \ref{sec:sec5.1.3} 


\subsection{Entrenamiento 3}\label{sec:sec5.2.3}

Este entrenamiento se ha realizado usando los pesos de clases descritos en el Cuadro \ref{tab:pesos_por_clase_entrenamiento}. Se ha sometido a 20 épocas y además de los callbacks de \textit{ModelCheckpoint} e \textit{HistorySaver}, se han usado también los callbacks de ReduceLROnPlateau y EarlyStopping.

Para mostrar los resultados de este entrenamiento se va a comenzar mostrando las métricas recogidas en el fichero generado por el callback \textit{HistorySaver} en la Figura \ref{fig:metrics_historic_mobilenetv2_1_3}.

\begin{figure}[ht]
\begin{center}
\includegraphics[scale=0.5]{./Images/Resultados/Histórico Métricas/mobilenetv2_1_3.png}
\caption{\headlinecolor{\underline{Histórico de métricas del entrenamiento 3 del modelo \textit{MobileNetV2} 1}}}
\label{fig:metrics_historic_mobilenetv2_1_3}
\end{center}
\end{figure}

En esta figura se puede observar el comportamiento de las métricas de precisión y pérdida de entrenamiento y validación a lo largo de las épocas a las que se ha sometido el modelo descrito en la subsección \ref{sec:sec4.4.1} en el entrenamiento. A continuación se describe lo que se puede interpretar:  

\begin{bulletlist}
\item Precisión de entrenamiento: Comienza alrededor de 0.65 y sube rápidamente a un valor cercano a 0.85, mostrando que el modelo aprende bien de los datos de entrenamiento.
\item Pérdida de entrenamiento: Disminuye de un valor cercano a 1.1 hasta cerca de 0.5, lo que confirma que el modelo está optimizando correctamente con la función de pérdida elegida para los datos de entrenamiento.
\item Precisión de validación: Aumenta de un valor cercano a 0.75 a un valor cercano a 0.85, lo que muestra que el modelo aprende bien de datos distintos a los de entrenamiento.
\item Pérdida de validación: Disminuye de un valor cercano a 0.8 a un valor cercano a 0.5, confirmando que la función de pérdida optimiza bien el modelo.
\end{bulletlist}

En resumen, ambas métricas de precisión tienden a subir acercándose a 1 y las métricas de pérdidas a bajar acercándose a 0. Ambas tendencias se realizan de manera progresiva sin señales de \textit{overfitting}. Cabe destacar que el callback de ReduceLROnPlateau se ha activado en la época 13 reduciendo el learning rate de \texttt{1e-3} a \texttt{5e-4} y en la época 18 reduciendo el learning rate de \texttt{5e-4} a \texttt{2.5e-4}, esxplicando la estabilización que existe.

Para poder comparar este entrenamiento con el resto es necesario realizar un análisis de métricas por clases. Para ello se van a mostrar las métricas del método \texttt{classification\_report} de la librería \textit{Sklearn} en la Figura \ref{fig:metrics_summary_mobilenetv2_1_3} y la matriz de confusión en la Figura \ref{fig:confusion_matrix_mobilenetv2_1_3}.

\begin{figure}[ht]
\begin{center}
\includegraphics[scale=0.7]{./Images/Resultados/Métricas resumen/mobilenetv2_1_3.png}
\caption{\headlinecolor{\underline{Resumen de métricas del modelo \textit{MobileNetV2} 1 en el entrenamiento 3}}}
\label{fig:metrics_summary_mobilenetv2_1_3}
\end{center}
\end{figure}

\begin{figure}[ht]
\begin{center}
\includegraphics[scale=0.5]{./Images/Resultados/Matriz de confusión/mobilenetv2_1_3.png}
\caption{\headlinecolor{\underline{Matriz de confusión del modelo \textit{MobileNetV2} 1 en el entrenamiento 3}}}
\label{fig:confusion_matrix_mobilenetv2_1_3}
\end{center}
\end{figure}

El modelo ha alcanzado una precisión del 83.1\% y un F1-score medio de 0.83, lo que indica un buen equilibrio entre precisión y capacidad de detección en las once clases evaluadas. Las clases \texttt{Tomato\_Yellow\_Leaf\_Curl\_Virus}, \texttt{Tomato\_mosaic\_virus} y \texttt{healthy} siguen destacando por sus altos valores de F1-score (superiores a 0,90), mostrando que el modelo identifica correctamente los patrones más distintivos de estas clases. Asimismo, el desempeño equilibrado entre las métricas macro y ponderadas (\textit{weighted}) sugiere que no existe un sesgo notable hacia clases con mayor cantidad de muestras.

Sin embargo, algunas clases como \texttt{Target\_Spot}, \texttt{Early\_blight} y \texttt{Septoria\_leaf\_spot} presentan valores más bajos, reflejando cierta confusión con otras enfermedades de síntomas visualmente similares.

La matriz de confusión de la Figura \ref{fig:confusion_matrix_mobilenetv2_1_3} refuerza las conclusiones obtenidas a partir de las métricas de evaluación, mostrando que el modelo clasifica correctamente la mayoría de las muestras en casi todas las clases. Se observa una fuerte concentración de valores en la diagonal principal, especialmente en las clases \texttt{Tomato\_Yellow\_Leaf\_Curl\_Virus}, \texttt{Tomato\_mosaic\_virus} y \texttt{healthy}. Estas clases presentan patrones visuales distintivos que el modelo ha aprendido de forma eficaz. 

En contraste, se aprecian errores recurrentes entre clases con síntomas similares, como \texttt{Early\_blight}, \texttt{Late\_blight}, \texttt{Septoria\_leaf\_spot} y \texttt{Target\_Spot} donde las confusiones cruzadas son más frecuentes. En conjunto, la matriz de confusión visualiza de manera clara la capacidad del modelo para distinguir la mayoría de las enfermedades, evidenciando a la vez los retos en la diferenciación de patologías con síntomas visualmente parecidos.

Este modelo tiene un comportamiento sólido y equilibrado, generaliza bien y consigue que incluso las clases con menos muestras tengan buenos niveles de precisión y sensibilidad. Para seguir refinando este modelo se descongelarán algunas capas del modelo base (\textit{fine tuning}) y se reducirá el \textit{learning rate}, estos cambos se aplican en el modelo de la Subsección \ref{sec:sec4.4.2}. 



\section{Modelo \textit{MobileNetV2} 2}\label{sec:sec5.3}

En esta sección se explicarán los resultados obtenidos al entrenar el modelo explicado en la Subsección \ref{sec:sec4.4.2}. El entrenamiento se ha realizado usando los pesos de clases descritos en el Cuadro \ref{tab:pesos_por_clase_entrenamiento}. Se ha sometido a 20 épocas y además de los callbacks de \textit{ModelCheckpoint} e \textit{HistorySaver}, se han usado también los callbacks de ReduceLROnPlateau y EarlyStopping.

Para mostrar los resultados de este entrenamiento se va a comenzar mostrando las métricas recogidas en el fichero generado por el callback \textit{HistorySaver} en la Figura \ref{fig:metrics_historic_mobilenetv2_2}.

\begin{figure}[ht]
\begin{center}
\includegraphics[scale=0.45]{./Images/Resultados/Histórico Métricas/mobilentv2_2.png}
\caption{\headlinecolor{\underline{Histórico de métricas del entrenamiento del modelo \textit{MobileNetV2} 2}}}
\label{fig:metrics_historic_mobilenetv2_2}
\end{center}
\end{figure}

En esta figura se puede observar el comportamiento de las métricas de precisión y pérdida de entrenamiento y validación a lo largo de las épocas a las que se ha sometido el modelo descrito en la subsección \ref{sec:sec4.4.2} en el entrenamiento. A continuación se describe lo que se puede interpretar:  

\begin{bulletlist}
\item Precisión de entrenamiento: Comienza alrededor de 0.5 y sube rápidamente a un valor cercano a 0.9, mostrando que el modelo aprende bien de los datos de entrenamiento.
\item Pérdida de entrenamiento: Disminuye de un valor cercano a 1.4 hasta cerca de 0.2, lo que confirma que el modelo está optimizando correctamente con la función de pérdida elegida para los datos de entrenamiento.
\item Precisión de validación: Aumenta de un valor próximo a 0.5 a un valor cercano a 0.9, lo que muestra que el modelo aprende bien de datos distintos a los de entrenamiento.
\item Pérdida de validación: Disminuye de un valor cercano a 1.4 a un valor próximo a 0.2, confirmando que la función de pérdida optimiza bien el modelo.
\end{bulletlist}

En resumen, ambas métricas de precisión tienden a subir acercándose a 1 y las métricas de pérdidas a bajar acercándose a 0. Ambas tendencias se realizan de manera progresiva sin señales de \textit{overfitting}. 

En términos generales se tiene un buen modelo, pero para poder comparar con el resto es necesario realizar un análisis de métricas por clases. Para ello se van a mostrar las métricas del método \texttt{classification\_report} de la librería \textit{Sklearn} en la Figura \ref{fig:metrics_summary_mobilenetv2_2} y la matriz de confusión en la Figura \ref{fig:confusion_matrix_mobilenetv2_2}.

\begin{figure}[ht]
\begin{center}
\includegraphics[scale=0.7]{./Images/Resultados/Métricas resumen/mobilentv2_2.png}
\caption{\headlinecolor{\underline{Resumen de métricas del modelo \textit{MobileNetV2} 2}}}
\label{fig:metrics_summary_mobilenetv2_2}
\end{center}
\end{figure}

\begin{figure}[ht]
\begin{center}
\includegraphics[scale=0.5]{./Images/Resultados/Matriz de confusión/mobilentv2_2.png}
\caption{\headlinecolor{\underline{Matriz de confusión del modelo \textit{MobileNetV2} 2}}}
\label{fig:confusion_matrix_mobilenetv2_2}
\end{center}
\end{figure}

El modelo ha alcanzado una precisión del 94\% y un F1-score medio de 0.93, lo que indica un buen equilibrio entre precisión y capacidad de detección en las once clases evaluadas. Las clases \texttt{Tomato\_Yellow\_Leaf\_Curl\_Virus}, \texttt{Tomato\_mosaic\_virus} y \texttt{healthy} siguen destacando por sus altos valores de F1-score (superiores a 0,95), mostrando que el modelo identifica correctamente los patrones más distintivos de estas clases. Asimismo, el desempeño equilibrado entre las métricas macro y ponderadas (\textit{weighted}) sugiere que no existe un sesgo notable hacia clases con mayor cantidad de muestras.

Sin embargo, algunas clases como \texttt{Target\_Spot} y \texttt{Early\_blight} presentan valores más bajos, reflejando cierta confusión con otras enfermedades de síntomas visualmente similares.

La matriz de confusión de la Figura \ref{fig:confusion_matrix_mobilenetv2_2} refuerza las conclusiones obtenidas a partir de las métricas de evaluación, mostrando que el modelo clasifica correctamente la mayoría de las muestras en casi todas las clases. Se observa una fuerte concentración de valores en la diagonal principal, especialmente en las clases \texttt{Tomato\_Yellow\_Leaf\_Curl\_Virus}, \texttt{Tomato\_mosaic\_virus} y \texttt{healthy}. Estas clases presentan patrones visuales distintivos que el modelo ha aprendido de forma eficaz. 

En contraste, se aprecian errores recurrentes entre clases con síntomas similares, como \texttt{Early\_blight}, \texttt{Septoria\_leaf\_spot} y \texttt{Target\_Spot} donde las confusiones cruzadas son más frecuentes. En conjunto, la matriz de confusión visualiza de manera clara la capacidad del modelo para distinguir la mayoría de las enfermedades, evidenciando a la vez los retos en la diferenciación de patologías con síntomas visualmente parecidos.

Este modelo tiene un comportamiento sólido y equilibrado, generaliza bien y consigue que incluso las clases con menos muestras tengan buenos niveles de precisión y sensibilidad. Para seguir refinando este modelo se descongelarán más capas del modelo base y se reducirá el \textit{learning rate}, estos cambos se aplican en el modelo de la Subsección \ref{sec:sec4.4.3}. 



\section{Modelo \textit{MobileNetV2} 3}\label{sec:sec5.4}

En esta sección se explicarán los resultados obtenidos al entrenar el modelo explicado en la Subsección \ref{sec:sec4.4.3}. El entrenamiento se ha realizado usando los pesos de clases descritos en el Cuadro \ref{tab:pesos_por_clase_entrenamiento}. Se ha sometido a 20 épocas y además de los callbacks de \textit{ModelCheckpoint} e \textit{HistorySaver}, se han usado también los callbacks de ReduceLROnPlateau y EarlyStopping.

Para mostrar los resultados de este entrenamiento se va a comenzar mostrando las métricas recogidas en el fichero generado por el callback \textit{HistorySaver} en la Figura \ref{fig:metrics_historic_mobilenetv2_2}.

\begin{figure}[ht]
\begin{center}
\includegraphics[scale=0.45]{./Images/Resultados/Histórico Métricas/mobilenetv2_3.png}
\caption{\headlinecolor{\underline{Histórico de métricas del entrenamiento del modelo \textit{MobileNetV2} 3}}}
\label{fig:metrics_historic_mobilenetv2_3}
\end{center}
\end{figure}

En esta figura se puede observar el comportamiento de las métricas de precisión y pérdida de entrenamiento y validación a lo largo de las épocas a las que se ha sometido el modelo descrito en la subsección \ref{sec:sec4.4.2} en el entrenamiento. A continuación se describe lo que se puede interpretar:  

\begin{bulletlist}
\item Precisión de entrenamiento: Comienza alrededor de 0.5 y sube rápidamente a un valor cercano a 0.9, mostrando que el modelo aprende bien de los datos de entrenamiento.
\item Pérdida de entrenamiento: Disminuye de un valor cercano a 1.75 hasta cerca de 0.25, lo que confirma que el modelo está optimizando correctamente con la función de pérdida elegida para los datos de entrenamiento.
\item Precisión de validación: Aumenta de un valor próximo a 0.5 a un valor cercano a 0.9, lo que muestra que el modelo aprende bien de datos distintos a los de entrenamiento.
\item Pérdida de validación: Disminuye de un valor cercano a 1.75 a un valor próximo a 0.25, confirmando que la función de pérdida optimiza bien el modelo.
\end{bulletlist}

En resumen, ambas métricas de precisión tienden a subir acercándose a 1 y las métricas de pérdidas a bajar acercándose a 0. Ambas tendencias se realizan de manera progresiva sin señales de \textit{overfitting}. 

En términos generales se tiene un buen modelo, pero para poder comparar con el resto es necesario realizar un análisis de métricas por clases. Para ello se van a mostrar las métricas del método \texttt{classification\_report} de la librería \textit{Sklearn} en la Figura \ref{fig:metrics_summary_mobilenetv2_3} y la matriz de confusión en la Figura \ref{fig:confusion_matrix_mobilenetv2_3}.

\begin{figure}[ht]
\begin{center}
\includegraphics[scale=0.7]{./Images/Resultados/Métricas resumen/mobilenetv2_3.png}
\caption{\headlinecolor{\underline{Resumen de métricas del modelo \textit{MobileNetV2} 3}}}
\label{fig:metrics_summary_mobilenetv2_3}
\end{center}
\end{figure}

\begin{figure}[ht]
\begin{center}
\includegraphics[scale=0.5]{./Images/Resultados/Matriz de confusión/mobilenetv2_3.png}
\caption{\headlinecolor{\underline{Matriz de confusión del modelo \textit{MobileNetV2} 3}}}
\label{fig:confusion_matrix_mobilenetv2_3}
\end{center}
\end{figure}

El modelo ha alcanzado una precisión del 92\% y un F1-score medio de 0.92, lo que indica un buen equilibrio entre precisión y capacidad de detección en las once clases evaluadas. Las clases \texttt{Tomato\_Yellow\_Leaf\_Curl\_Virus}, \texttt{Tomato\_mosaic\_virus} y \texttt{healthy} siguen destacando por sus altos valores de F1-score (superiores a 0,95), mostrando que el modelo identifica correctamente los patrones más distintivos de estas clases. Asimismo, el desempeño equilibrado entre las métricas macro y ponderadas (\textit{weighted}) sugiere que no existe un sesgo notable hacia clases con mayor cantidad de muestras.

Sin embargo, algunas clases como \texttt{Target\_Spot} y \texttt{Septoria\_leaf\_spot} presentan valores más bajos, reflejando cierta confusión con otras enfermedades de síntomas visualmente similares.

La matriz de confusión de la Figura \ref{fig:confusion_matrix_mobilenetv2_3} refuerza las conclusiones obtenidas a partir de las métricas de evaluación, mostrando que el modelo clasifica correctamente la mayoría de las muestras en casi todas las clases. Se observa una fuerte concentración de valores en la diagonal principal, especialmente en las clases \texttt{Tomato\_Yellow\_Leaf\_Curl\_Virus}, \texttt{Tomato\_mosaic\_virus} y \texttt{healthy}. Estas clases presentan patrones visuales distintivos que el modelo ha aprendido de forma eficaz. 

En contraste, se aprecian errores recurrentes entre clases con síntomas similares, como \texttt{Septoria\_leaf\_spot} y \texttt{Target\_Spot} donde las confusiones cruzadas son más frecuentes. En conjunto, la matriz de confusión visualiza de manera clara la capacidad del modelo para distinguir la mayoría de las enfermedades, evidenciando a la vez los retos en la diferenciación de patologías con síntomas visualmente parecidos.

Este modelo tiene un comportamiento sólido y equilibrado, generaliza bien y consigue que incluso las clases con menos muestras tengan buenos niveles de precisión y sensibilidad. 

En este modelo se han desbloquedo más capas del modelo base dando más libertad para aprender características específicas (como texturas finas de enfermedades). Pero también rompe parcialmente la estructura previa de ImageNet, reduciendo la estabilidad de las características más generales. Eso demuestra que se observen mejoras en alguna clase, pero pérdida de rendimiento general.



\section{Modelo \textit{EfficientNetB0} 1}\label{sec:sec5.5}

En esta sección se van a mostrar los resultados del entrenamiento del modelo descrito en la subsección \ref{sec:sec4.4.4}. Este entrenamiento se ha realizado usando los pesos de clases descritos en el Cuadro \ref{tab:pesos_por_clase_entrenamiento}. Solo se ha sometido a 10 épocas y se han usado los callbacks de \textit{ModelCheckpoint} e \textit{HistorySaver}, ya que el objetivo de este entrenamiento es observar la tendencia inicial.

Para mostrar los resultados de este entrenamiento se va a comenzar mostrando las métricas recogidas en el fichero generado por el callback \textit{HistorySaver} en la Figura \ref{fig:summary_metrics_efficientnetb0_1}.

\begin{figure}[ht]
\begin{center}
\includegraphics[scale=0.5]{./Images/Resultados/Histórico Métricas/efficientnetb0_1.png}
\caption{\headlinecolor{\underline{Histórico de métricas del del modelo \textit{EfficientNetB0} 1}}}
\label{fig:summary_metrics_efficientnetb0_1}
\end{center}
\end{figure}

En esta figura se puede observar el comportamiento de las métricas de precisión y pérdida de entrenamiento y validación a lo largo de las épocas a las que se ha sometido el modelo descrito en la subsección \ref{sec:sec4.4.4} en el entrenamiento. A continuación se describe lo que se puede interpretar:  

\begin{bulletlist}
\item Precisión de entrenamiento: Queda constante en aproximadamente 0.09, un valor muy lejano a 1, lo que demuestra que el modelo no aprende de los datos de entrenamiento.
\item Pérdida de entrenamiento: Queda practicamente constante en un valor cercano a 2.4, bastante lejano a 0.
\item Precisión de validación: Oscila entre valores cercanos a 0.1, por lo que el modelo tampoco generaliza.
\item Pérdida de validación: Oscila entre valores cercanos a 2.4, bastante lejano a 0.
\end{bulletlist}

El modelo no está aprendiendo absolutamente nada más allá del azar. Para mejorar esta situación se descongelarán algunas capas superiores del modelo base \textit{EfficientNetB0}, haciendo uso de la técnica de \textit{fine tuning}.


\section{Modelo \textit{EfficientNetB0} 2}\label{sec:sec5.6}

En esta sección se van a mostrar los resultados del entrenamiento del modelo descrito en la subsección \ref{sec:sec4.4.5}. Este entrenamiento se ha realizado usando los pesos de clases descritos en el Cuadro \ref{tab:pesos_por_clase_entrenamiento}. Solo se ha sometido a 10 épocas y se han usado los callbacks de \textit{ModelCheckpoint} e \textit{HistorySaver}, ya que el objetivo de este entrenamiento es observar la tendencia inicial.

Para mostrar los resultados de este entrenamiento se va a comenzar mostrando las métricas recogidas en el fichero generado por el callback \textit{HistorySaver} en la Figura \ref{fig:summary_metrics_efficientnetb0_2}.

\begin{figure}[ht]
\begin{center}
\includegraphics[scale=0.45]{./Images/Resultados/Histórico Métricas/efficientnetb0_2.png}
\caption{\headlinecolor{\underline{Histórico de métricas del entrenamiento del modelo \textit{EfficientNetB0} 2}}}
\label{fig:summary_metrics_efficientnetb0_2}
\end{center}
\end{figure}

En esta figura se puede observar el comportamiento de las métricas de precisión y pérdida de entrenamiento y validación a lo largo de las épocas a las que se ha sometido el modelo descrito en la subsección \ref{sec:sec4.4.5} en el entrenamiento. A continuación se describe lo que se puede interpretar:  

\begin{bulletlist}
\item Precisión de entrenamiento: Aumenta desde un valor aproximado a 0.2 a un valor aproximado a 0.5, lo que demuestra que el modelo está aprendiendo de los datos de entrenamiento.
\item Pérdida de entrenamiento: Disminuye de un valor cercano a 2.2 a un valor aproximado a 1.4, mostrando una tendencia decreciente.
\item Precisión de validación: Decrece de un valor cercano a 0.25 a un valor cercano a 0.15, por lo que el modelo no generaliza.
\item Pérdida de validación: Aumenta desde un valor aproximado a 2.2 a un valor aproximado a 2.8, bastante lejano a 0.
\end{bulletlist}

En definitiva, se puede decir que el modelo está aprendiendo a memorizar el conjunto de entrenamiento, pero pierde capacidad de generalización, es decir tiene sobreajuste. Para mejorar esta situación, se va a disminuir el \textit{learning rate} en el modelo de la sección \ref{sec:sec4.4.6}


\section{Modelo \textit{EfficientNetB0} 3}\label{sec:sec5.7}

En esta sección se van a mostrar los resultados del entrenamiento del modelo descrito en la subsección \ref{sec:sec4.4.6}. Este entrenamiento se ha realizado usando los pesos de clases descritos en el Cuadro \ref{tab:pesos_por_clase_entrenamiento}. Solo se ha sometido a 10 épocas y se han usado los callbacks de \textit{ModelCheckpoint} e \textit{HistorySaver}, ya que el objetivo de este entrenamiento es observar la tendencia inicial.

Para mostrar los resultados de este entrenamiento se va a comenzar mostrando las métricas recogidas en el fichero generado por el callback \textit{HistorySaver} en la Figura \ref{fig:summary_metrics_efficientnetb0_3}.

\begin{figure}[ht]
\begin{center}
\includegraphics[scale=0.5]{./Images/Resultados/Histórico Métricas/efficientnetb0_3.png}
\caption{\headlinecolor{\underline{Histórico de métricas del entrenamiento del modelo \textit{EfficientNetB0} 1}}}
\label{fig:summary_metrics_efficientnetb0_3}
\end{center}
\end{figure}

En esta figura se puede observar el comportamiento de las métricas de precisión y pérdida de entrenamiento y validación a lo largo de las épocas a las que se ha sometido el modelo descrito en la subsección \ref{sec:sec4.4.6} en el entrenamiento. A continuación se describe lo que se puede interpretar:  

\begin{bulletlist}
\item Precisión de entrenamiento: Aumenta desde un valor aproximado a 0.15 a un valor aproximado a 0.35, lo que demuestra que el modelo está aprendiendo de los datos de entrenamiento.
\item Pérdida de entrenamiento: Disminuye de un valor cercano a 2.3 a un valor aproximado a 1.8, mostrando una tendencia descendente.
\item Precisión de validación: Aumenta de un valor cercano a 0.15 a un valor cercano a 0.25, por lo que el está empezando a generalizar.
\item Pérdida de validación: Oscila en un valor cercano a 2.2, por lo que no existe un generalización clara.
\end{bulletlist}

En definitiva, se puede decir que se ha disminuido el sobreajuste, el modelo ha comenzado a generalizar, pero no lo suficiente y con la suficiente estabilidad para considerarlo un buen resultado. Para afinar más este modelo se podría incluir una capa de \textit{dropout} antes de la capa densa y entrenar más épocas con los callbacks de ReduceLROnPlateau y EarlyStopping. Aún así, no se esperan excelentes resultados, simplemente una mejora en la regularización.


\section{Modelo \textit{NASNetMobile} 1}\label{sec:sec5.8}

En esta sección se van a mostrar los resultados del entrenamiento del modelo descrito en la subsección \ref{sec:sec4.4.7}. Este entrenamiento se ha realizado usando los pesos de clases descritos en el Cuadro \ref{tab:pesos_por_clase_entrenamiento}.Solo see ha sometido a 15 épocas y se han usado los callbacks de \textit{ModelCheckpoint} e \textit{HistorySaver}, ya que el objetivo de este entrenamiento es observar la tendencia inicial.

Para mostrar los resultados de este entrenamiento se va a comenzar mostrando las métricas recogidas en el fichero generado por el callback \textit{HistorySaver} en la Figura \ref{fig:metrics_historic_nasnetmobile_1}.

\begin{figure}[ht]
\begin{center}
\includegraphics[scale=0.5]{./Images/Resultados/Histórico Métricas/nasnetmobile_1.png}
\caption{\headlinecolor{\underline{Histórico de métricas del entrenamiento del modelo \textit{NASNetMobile} 1}}}
\label{fig:metrics_historic_nasnetmobile_1}
\end{center}
\end{figure}

En esta figura se puede observar el comportamiento de las métricas de precisión y pérdida de entrenamiento y validación a lo largo de las épocas a las que se ha sometido el modelo descrito en la subsección \ref{sec:sec4.4.1} en el entrenamiento. A continuación se describe lo que se puede interpretar:  

\begin{bulletlist}
\item Precisión de entrenamiento: Comienza alrededor de 0.5 y sube rápidamente a un valor cercano a 0.7, mostrando que el modelo aprende bien de los datos de entrenamiento.
\item Pérdida de entrenamiento: Disminuye de un valor cercano a 1.3 hasta cerca de 0.7, lo que confirma que el modelo está optimizando correctamente con la función de pérdida elegida para los datos de entrenamiento.
\item Precisión de validación: Aumenta de un valor cercano a 0.65 a un valor cercano a 0.75, lo que muestra que el modelo aprende bien de datos distintos a los de entrenamiento.
\item Pérdida de validación: Disminuye de un valor cercano a 1.1 a un valor cercano a 0.8, confirmando que la función de pérdida optimiza bien el modelo.
\end{bulletlist}

El modelo mejora rápidamente durante las primeras 5 épocas. A partir de ese punto, tanto la precisión de entrenmaiento como la de validación se estabilizan alrededor del 75\%, lo cual indica que el modelo ha aprendido bien las características generales, pero está alcanzando su límite. 

Para mejorar este resultado se podría entrenar más épocas añadiendo los callbacks \textit{EarlyStopping} para que pare la ejecución si el modelo no mejora y con \textit{ReduceLROnPlateau} para que disminuya el \textit{learning rate} en el entrenamiento y haga que no se estanquen sus valores.



\section{Modelo \textit{NASNetMobile} 2}\label{sec:sec5.9}

En esta sección se van a mostrar los resultados del entrenamiento del modelo descrito en la subsección \ref{sec:sec4.4.8}. Este entrenamiento se ha realizado usando los pesos de clases descritos en el Cuadro \ref{tab:pesos_por_clase_entrenamiento}. Se ha sometido a 10 épocas y se han usado los callbacks de \textit{ModelCheckpoint} e \textit{HistorySaver}.

Para mostrar los resultados del mismo se va a comenzar mostrando las métricas recogidas en el fichero generado por el callback \textit{HistorySaver} en la Figura \ref{fig:metrics_historic_nasnetmobile_2}.

\begin{figure}[ht]
\begin{center}
\includegraphics[scale=0.45]{./Images/Resultados/Histórico Métricas/nasnetmobile_2.png}
\caption{\headlinecolor{\underline{Histórico de métricas del entrenamiento del modelo \textit{NASNetMobile} 2}}}
\label{fig:metrics_historic_nasnetmobile_2}
\end{center}
\end{figure}

En esta figura se puede observar el comportamiento de las métricas de precisión y pérdida de entrenamiento y validación a lo largo de las épocas a las que se ha sometido el modelo descrito en la subsección \ref{sec:sec4.4.8} en el entrenamiento. A continuación se describe lo que se puede interpretar:  

\begin{bulletlist}
\item Precisión de entrenamiento: Comienza alrededor de 0.2 y sube rápidamente a un valor cercano a 0.9, mostrando que el modelo aprende bien de los datos de entrenamiento.
\item Pérdida de entrenamiento: Disminuye de un valor cercano a 2 hasta cerca de 0.25, lo que confirma que el modelo está optimizando correctamente con la función de pérdida elegida para los datos de entrenamiento.
\item Precisión de validación: Aumenta de un valor próximo a 0.2 a un valor cercano a 0.9, lo que muestra que el modelo aprende bien de datos distintos a los de entrenamiento.
\item Pérdida de validación: Disminuye de un valor cercano a 2 a un valor próximo a 0.25, confirmando que la función de pérdida optimiza bien el modelo.
\end{bulletlist}


En resumen, ambas métricas de precisión tienden a subir acercándose a 1 y las métricas de pérdidas a bajar acercándose a 0. Ambas tendencias se realizan de manera progresiva sin señales de \textit{overfitting}. 

En términos generales se tiene un buen modelo, pero para poder comparar con el resto es necesario realizar un análisis de métricas por clases. Para ello se van a mostrar las métricas del método \texttt{classification\_report} de la librería \textit{Sklearn} en la Figura \ref{fig:metrics_summary_nasnetmobile_2} y la matriz de confusión en la Figura \ref{fig:confusion_matrix_nasnetmobile_2}.

\begin{figure}[ht]
\begin{center}
\includegraphics[scale=0.7]{./Images/Resultados/Métricas resumen/nasnetmobile_2.png}
\caption{\headlinecolor{\underline{Resumen de métricas del modelo \textit{NASNetMobile} 2}}}
\label{fig:metrics_summary_nasnetmobile_2}
\end{center}
\end{figure}

\begin{figure}[ht]
\begin{center}
\includegraphics[scale=0.5]{./Images/Resultados/Matriz de confusión/nasnetmobile_2.png}
\caption{\headlinecolor{\underline{Matriz de confusión del modelo \textit{NASNetMobile} 2}}}
\label{fig:confusion_matrix_nasnetmobile_2}
\end{center}
\end{figure}


El modelo ha alcanzado una precisión del 92\% y un F1-score medio de 0.92, lo que indica un buen equilibrio entre precisión y capacidad de detección en las once clases evaluadas. Las clases \texttt{Tomato\_Yellow\_Leaf\_Curl\_Virus}, \texttt{Tomato\_mosaic\_virus} y \texttt{healthy} siguen destacando por sus altos valores de F1-score (superiores a 0,95), mostrando que el modelo identifica correctamente los patrones más distintivos de estas clases. Asimismo, el desempeño equilibrado entre las métricas macro y ponderadas (\textit{weighted}) sugiere que no existe un sesgo notable hacia clases con mayor cantidad de muestras.

Sin embargo, algunas clases como \texttt{Septoria\_leaf\_spot}, \texttt{Early\_blight} y \texttt{Late\_blight} presentan valores más bajos, reflejando cierta confusión con otras enfermedades de síntomas visualmente similares.

La matriz de confusión de la Figura \ref{fig:confusion_matrix_nasnetmobile_2} refuerza las conclusiones obtenidas a partir de las métricas de evaluación, mostrando que el modelo clasifica correctamente la mayoría de las muestras en casi todas las clases. Se observa una fuerte concentración de valores en la diagonal principal, especialmente en las clases \texttt{Tomato\_Yellow\_Leaf\_Curl\_Virus}, \texttt{Tomato\_mosaic\_virus} y \texttt{healthy}. Estas clases presentan patrones visuales distintivos que el modelo ha aprendido de forma eficaz. 

En contraste, se aprecian errores recurrentes entre clases con síntomas similares, como \texttt{Septoria\_leaf\_spot}, \texttt{Early\_blight} y \texttt{Late\_blight} donde las confusiones cruzadas son más frecuentes. En conjunto, la matriz de confusión visualiza de manera clara la capacidad del modelo para distinguir la mayoría de las enfermedades, evidenciando a la vez los retos en la diferenciación de patologías con síntomas visualmente parecidos.

Este modelo tiene un comportamiento sólido y equilibrado, generaliza bien y consigue que incluso las clases con menos muestras tengan buenos niveles de precisión y sensibilidad. 

Descongelar desde la capa 60 hacia adelante ha sido una excelente elección. Se ha permitido que los bloques finales aprendan patrones específicos de las hojas y texturas de las enfermedades y el \textit{learning rate} bajo fue clave para no destruir los pesos de \textit{ImageNet}.



\section{Comparativa de modelos}\label{sec:sec5.10}

En esta sección se van a comparar los mejores modelos que se han hallado en este estudio, es decir, se van a comparar los modelos de las subsecciones \ref{sec:sec4.4.2}, \ref{sec:sec4.4.3} y \ref{sec:sec4.4.8}, a partir de los resultados obtenidos en las secciones \ref{sec:sec5.2}, \ref{sec:sec5.3} y \ref{sec:sec5.8} respectivamente. Para ello, en el Cuadro \ref{tab:resumen_modelos} se presentan las características generales de los modelos junto con un número asociado en la columna "Número", que será usado como identificador del modelo en el Cuadro \ref{tab:comparativa_modelos}, donde se tienen las principales métricas de cada modelo.

\begin{table}[ht] 
\centering
\caption{\headlinecolor{\underline{Resúmenes de modelos}}}
\label{tab:resumen_modelos}
\begin{tabular}{|c|c|c|c|c|} \hline
\rowcolor{naranja}
\multicolumn{5}{|c|}{\textbf{Resumen de modelos}} \\ \hline
\rowcolor{naranja!30}
Modelo & Número & Fine tuning & Learning rate & Épocas \\ \hline
MobileNetV2 2 & 1 & 100 capas & \texttt{1e-5} & 20 \\ \hline
MobileNetV2 3 & 2 & 80 capas & \texttt{5e-6} & 20 \\ \hline
NASNetMobile 2 & 3 & 60 capas & \texttt{5e-6} & 15 \\ \hline
\end{tabular}
\end{table}


\begin{table}[ht] 
\centering
\caption{\headlinecolor{\underline{Comparativa de modelos con métricas globales}}}
\label{tab:comparativa_modelos}
\begin{tabularx}{\textwidth}{|>{\centering\arraybackslash}p{1.15cm}|Y|>{\centering\arraybackslash}p{1.9cm}|>{\centering\arraybackslash}p{2cm}|>{\centering\arraybackslash}p{2cm}|>{\centering\arraybackslash}p{1.6cm}|>{\centering\arraybackslash}p{1.9cm}|} \hline
\rowcolor{naranja}
\multicolumn{7}{|c|}{\textbf{Resumen de métricas}} \\ \hline
\rowcolor{naranja!30}
Modelo & Precisión macro & Precisión ponderada & Sensibilidad macro & Sensibilidad ponderada & F1-Score macro & F1-Score ponderada \\ \hline
1 & 0.940 & 0.941 & 0.941 & 0.939 & 0.939 & 0.939 \\ \hline
2 & 0.925 & 0.927 & 0.927 & 0.924 & 0.924 & 0.924 \\ \hline
3 & 0.925 & 0.923 & 0.926 & 0.920 & 0.925 & 0.920 \\ \hline
\end{tabularx}
\end{table}

Los tres modelos presentan rendimientos muy próximos, con ambas precisiones superiores al 92\%, lo que indica una buena capacidad de generalización en la clasificación de las once clases.También cabe destacar que los tres modelos en las tres métricas tienen un valor muy parecido en macro y weighted, lo que sugiere que el conjunto de datos está relativamente equilibrado entre clases, y que los modelos no presentan sesgos fuertes hacia ninguna clase.

El modelo MobileNetV2 2 obtiene las mejores métricas globales, alcanzando una precisión del 94\% y un f1-score de 93.9\%. Este modelo destaca por su rendimiento equilibrado en la mayoría de las clases, con especial solidez en \texttt{Tomato\_Yellow\_Leaf\_Curl\_Virus} (f1 = 0.985), \texttt{Tomato\_mosaic\_virus} (f1 = 0.974) y \texttt{healthy} (f1 = 0.968).

En comparación, el modelo MobileNetV2 3 muestra un rendimiento ligeramente inferior con una precisión de 92.4\%, aunque mantiene una buena estabilidad entre clases. Las diferencias más notables respecto al modelo anterior se observan en \texttt{Septoria\_leaf\_spot} y \texttt{Spidermite\_Two-spotted\_spider\_mite}, donde presenta valores de sensibilidad algo más bajos.

Por su parte, NASNetMobile 2 alcanza un rendimiento global similar con una precisión de 92.0\%, con un comportamiento especialmente bueno en las clases \texttt{Spidermite\_Two-spotted\_spider\_mite} (f1 = 0.962) y \texttt{Target\_spot} (f1 = 0.931). Sin embargo, en otras clases como \texttt{Bacterial\_spot} o \texttt{Septoria\_leaf\_spot} su precisión se reduce notablemente, lo que sugiere una menor estabilidad interclase.

En conjunto, los resultados confirman que las arquitecturas basadas en MobileNetV2 ofrecen un mejor resultado. Las diferencias entre los modelos son moderadas, pero la configuración con fine tuning más profundo (100 capas) muestra una ligera ventaja consistente en las métricas promedio y una mayor estabilidad entre clases.

Estos resultados serán considerados en el apartado \ref{cap:cap6} para la selección del modelo final.



\chapter{Conclusiones}\label{cap:cap6}

Las conclusiones obtenidas tras el desarrollo completo de este proyecto son las siguientes:

\begin{enumerate}[label=\bfseries\headlinecolor\arabic*.]
\item Se realizó una revisión de la literatura sobre estudios relacionados con la detección de enfermedades en plantas, lo que permitió seleccionar arquitecturas preentrenadas ligeras y eficientes, como \textit{MobileNetV2} y \textit{NASNetMobile}.
\item Se analizaron los datos disponibles y se realizaron distintas tareas como adaptación de número de píxeles por imagen, normalización, creación de conjunto de pruebas, \textit{data augmentation} y balanceo de clases con pesos predefinidos (\textit{class weight}),  para asegurar su homogeneidad, además de una estructuración equilibrada entre clases que facilitó un entrenamiento estable.
\item Se desarrollaron ocho modelos basados en arquitecturas (MobileNetV2, EfficientNetB0, NASNetMobile) preentrenadas con ImageNet, ajustando parámetros como el \textit{learning rate} y el número de capas finamente ajustadas (\textit{fine tuning}).
\item Se realizaron once entrenamientos, de los cuales tres modelos destacaron por su rendimiento. La comparativa detallada mostró que MobileNetV2 con \textit{fine tuning} de 100 capas y learning rate de \texttt{1e-5} alcanzó los mejores resultados globales, con una accuracy del 93.9\% y un f1-score macro de 0.939, superando ligeramente al resto de configuraciones y mostrando una mayor estabilidad interclase.
\end{enumerate}

En consecuencia, se selecciona este modelo (\ref{sec:sec4.4.2}) como el modelo final propuesto, al ofrecer el mejor equilibrio entre métricas globales e interclase, que garantizan una mayor precisión y capacidad de generalización. Este modelo cumple el objetivo de desarrollar un sistema preciso y adaptable para la clasificación automática de enfermedades en hojas de tomate. Los resultados obtenidos permiten afirmar que dicho objetivo se ha alcanzado satisfactoriamente, logrando un modelo con un rendimiento competitivo y una adecuada robustez frente a la variabilidad del conjunto de datos.

Como líneas futuras de trabajo, se plantea incorporar técnicas \textit{ensembles} que permita combinar las diferentes arquitecturas para lograr mejorar la detección de las clases más conflictivas \parencite{}.


\printbibliography

\end{document}
